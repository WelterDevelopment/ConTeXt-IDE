\environment leaflet-common

\startdocument[graphic=2]

\startbuffer[1]
    The \LUAMETATEX\ engine is a follow up on \LUATEX. It integrates the \TEX\
    text rendering engine, the \METAPOST\ graphic engine and the \LUA\ script
    interpreter. The development is part of the \CONTEXT\ macro package
    development. This macro package tightly integrates the three subsystems. The
    \LUAMETATEX\ code is part of the \CONTEXT\ distribution.
\stopbuffer

\startbuffer[2]
    The \LUAMETATEX\ engine is lean and mean. There is for instance no backend
    code present. In \CONTEXT\ this is handled in \LUA. Graphic inclusion is also
    delegated to \LUA. The \TEX\ frontend is a slightly extended version of the
    \LUATEX\ one. System depedencies are minimized. Where possible we stay close
    to the original \TEX\ concept because that is a well documented reference.
    The binary can also be used as stand alone \LUA\ engine.
\stopbuffer

\startbuffer[3]
    The \METAPOST\ library also has access to \LUA, which means that the language
    can be enhanced and functionality added on demand. There are several graphic
    libraries provided in \CONTEXT. This graphical language is efficient in
    runtime and graphical output. In combination with \LUA\ we have a high
    performance graphical subsystem that can handle a huge amount of data.
    Additional text (like labels) is typeset at high quality.
\stopbuffer

\startbuffer[4]
    The \LUA\ code that comes with \CONTEXT\ contains a lot of helper code which
    means that one can set up selfcontained workflows without many extra
    dependencies. Documents can be encoded in \TEX, \LUA, \XML\ or whatever
    suits. There is support for databases too.
\stopbuffer

\startbuffer[5]
    The \CONTEXT\ code base evolved over time. The basic functionality is quite stable.
    The move from \MKII\ to \MKIV\ to \LMTX\ has been gradual. The efficiency in terms
    of code and performance has been improved stepwise. Development continues and beta
    releases occur on a regular basis. The \CONTEXT\ user community is quite willing
    to experiment with betas that can be installed alongside stable snapshots.

    \blank

    The installation is relatively small (fonts make up much of it) and updating is
    easy. We operate in the \TEX\ Directory Structure, which is a proven concept.
\stopbuffer

\startpagemakeup
    \offinterlineskip
    \vskip24pt
    \hbox to \hsize \bgroup
        \hss
        \startframed[align=normal,width=.7\textwidth,frame=off]
            \getbuffer[1]
        \stopframed
        \hskip1cm
    \egroup
    \vfill
    \hbox to \hsize \bgroup
        \hskip1cm
        \startframed[align=normal,width=.7\textwidth,frame=off]
            \getbuffer[2]
        \stopframed
    \egroup
    \vfill
    \hbox to \hsize \bgroup
        \hss
        \startframed[align=normal,width=.7\textwidth,frame=off]
            \getbuffer[3]
        \stopframed
        \hskip1cm
    \egroup
    \vfill
    \hbox to \hsize \bgroup
        \hss
        \setupitemize[after=]
        \startframed[align=normal,width=\dimexpr\textwidth-2cm\relax,frame=off]
            \getbuffer[4]
        \stopframed
        \hskip1cm
    \egroup
    \vfill
    \hbox to \hsize \bgroup
        \hss
        \startframed[align=normal,width=.7\textwidth,frame=off]
            \getbuffer[5]
        \stopframed
        \hskip1cm
    \egroup
    \vfill
    \vfill
    \vfill
    \hbox to \hsize \bgroup
        \hskip1cm
        \startframed[align=normal,width=.7\textwidth,frame=off]
            \bfd \LUAMETATEX \enspace \emdash \enspace factsheet
        \stopframed
        \hss
    \egroup
    \vskip12pt
\stoppagemakeup

\stopdocument
