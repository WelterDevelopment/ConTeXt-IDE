% language=uk

\environment luametatex-style

\startcomponent luametatex-differences

\startchapter[reference=differences,title={Differences with \LUATEX}]

As \LUAMETATEX\ is a leaner and meaner \LUATEX, this chapter will discuss
what is gone. We start with the primitives that were dropped.

\starttabulate[|l|pl|]
\NC fonts       \NC \type {\letterspacefont}
                    \type {\copyfont}
                    \type {\expandglyphsinfont}
                    \type {\ignoreligaturesinfont}
                    \type {\tagcode}
                    \type {\leftghost}
                    \type {\rightghost}
                \NC \NR
\NC backend     \NC \type {\dviextension}
                    \type {\dvivariable }
                    \type {\dvifeedback}
                    \type {\pdfextension}
                    \type {\pdfvariable }
                    \type {\pdffeedback}
                    \type {\dviextension}
                    \type {\draftmode}
                    \type {\outputmode}
                \NC \NR
\NC dimensions  \NC \type {\pageleftoffset}
                    \type {\pagerightoffset}
                    \type {\pagetopoffset}
                    \type {\pagebottomoffset}
                    \type {\pageheight}
                    \type {\pagewidth}
                \NC \NR
\NC resources   \NC \type {\saveboxresource}
                    \type {\useboxresource}
                    \type {\lastsavedboxresourceindex}
                    \type {\saveimageresource}
                    \type {\useimageresource}
                    \type {\lastsavedimageresourceindex}
                    \type {\lastsavedimageresourcepages}
                \NC \NR
\NC positioning \NC \type {\savepos}
                    \type {\lastxpos}
                    \type {\lastypos}
                \NC \NR
\NC directions  \NC \type {\textdir}
                    \type {\linedir}
                    \type {\mathdir}
                    \type {\pardir}
                    \type {\pagedir}
                    \type {\bodydir}
                    \type {\pagedirection}
                    \type {\bodydirection}
                \NC \NR
\NC randomizer  \NC \type {\randomseed}
                    \type {\setrandomseed}
                    \type {\normaldeviate}
                    \type {\uniformdeviate}
                \NC \NR
\NC utilities   \NC \type {\synctex}
                \NC \NR
\NC extensions  \NC \type {\latelua}
                    \type {\lateluafunction}
                    \type {\immediate}
                    \type {\openout}
                    \type {\write}
                    \type {\closeout}
                \NC \NR
\NC control     \NC \type {\suppressfontnotfounderror}
                    \type {\suppresslongerror}
                    \type {\suppressprimitiveerror}
                    \type {\suppressmathparerror}
                    \type {\suppressifcsnameerror}
                    \type {\suppressoutererror}
                    \type {\mathoption}
                \NC \NR
\NC whatever    \NC \type {\primitive}
                    \type {\ifprimitive}
                \NC \NR
\NC ignored     \NC \type {\long}
                    \type {\outer}
                    \type {\mag}
                \NC \NR
\stoptabulate

The resources and positioning primitives are actually useful but can be defined
as macros that (via \LUA) inject nodes in the input that suit the macro package
and backend. The three||letter direction primitives are gone and the numeric
variants are now leading. There is no need for page and body related directions
and they don't work well in \LUATEX\ anyway. We only have two directions left.

The primitive related extensions were not that useful and reliable so they have
been removed. There are some new variants that will be discussed later. The \type
{\outer} and \type {\long} prefixes are gone as they don't make much sense
nowadays and them becoming dummies opened the way to something new, again to be
discussed elsewhere. I don't think that (\CONTEXT) users will notice it. The
\type {\suppress..} features are now default.

The \type {\shipout} primitive does no ship out but just erases the content of
the box, if that hasn't happened already in another way.

The extension primitives relate to the backend (when not immediate) and can be
implemented as part of a backend design using generic whatsits. There is only one
type of whatsit now. In fact we're now closer to original \TEX\ with respect to
the extensions.

The \type {img} library has been removed as it's rather bound to the backend. The
\type {slunicode} library is also gone. There are some helpers in the string
library that can be used instead and one can write additional \LUA\ code if
needed. There is no longer a \type {pdf} backend library.

In the \type {node}, \type {tex} and \type {status} library we no longer have
helpers and variables that relate to the backend. The \LUAMETATEX\ engine is in
principle \DVI\ and \PDF\ unaware. There are only generic whatsit nodes that can
be used for some management related tasks. For instance you can use them to
implement user nodes.

The margin kern nodes are gone and we now use regular kern nodes for them. As a
consequence there are two extra subtypes indicating the injected left or right
kern. The glyph field served no real purpose so there was no reason for a special
kind of node.

The \KPSE\ library is no longer built|-|in. Because there is no backend, quite
some file related callbacks could go away. The following file related callbacks
remained (till now):

\starttyping
find_write_file find_data_file find_format_file
open_data_file read_data_file
\stoptyping

Also callbacks related to errors stay:

\starttyping
show_error_hook show_lua_error_hook,
show_error_message show_warning_message
\stoptyping

The (job) management hooks are kept:

\starttyping
process_jobname
start_run stop_run wrapup_run
pre_dump
start_file stop_file
\stoptyping

Because we use a more generic whatsit model, there is a new callback:

\starttyping
show_whatsit
\stoptyping

Being the core of extensibility, the typesetting callbacks of course stayed. This
is what we ended up with:

% \ctxlua{inspect(table.sortedkeys(callbacks.list))}

\starttyping
find_log_file, find_data_file, find_format_file, open_data_file, read_data_file,
process_jobname, start_run, stop_run, define_font, pre_output_filter,
buildpage_filter, hpack_filter, vpack_filter, hyphenate, ligaturing, kerning,
pre_linebreak_filter, linebreak_filter, post_linebreak_filter,
append_to_vlist_filter, mlist_to_hlist, pre_dump, start_file, stop_file,
handle_error_hook, show_error_hook, show_lua_error_hook, show_error_message,
show_warning_message, hpack_quality, vpack_quality, insert_local_par,
contribute_filter, build_page_insert, wrapup_run, new_graf, make_extensible,
show_whatsit, terminal_input,
\stoptyping

As in \LUATEX\ font loading happens with the following callback. This time it
really needs to be set because there is no built|-|in font loader.

\starttyping
define_font
\stoptyping

There are all kinds of subtle differences in the implementation, for instance we
no longer intercept \type {*} and \type {&} as these were already replaced long
ago in \TEX\ engines by command line options. Talking of options, only a few are
left.

We took our time for reaching a stable state in \LUATEX. Among the reasons is the
fact that most was experimented with in \CONTEXT. It took many man|-|years to
decide what to keep and how to do things. Of course there are places when things
can be improved and it might happen in \LUAMETATEX. Contrary to what is sometimes
suggested, the \LUATEX|-|\CONTEXT\ \MKIV\ combination (assuming matched versions)
has been quite stable. It made no sense otherwise. Most \CONTEXT\ functionality
didn't change much at the user level. Of course there have been issues, as is
natural with everything new and beta, but we have a fast update cycle.

The same is true for \LUAMETATEX\ and \CONTEXT\ \LMTX: it can be used for
production as usual and in practice \CONTEXT\ users tend to use the beta
releases, which proves this. Of course, if you use low level features that are
experimental you're on your own. Also, as with \LUATEX\ it might take many years
before a long term stable is defined. The good news is that, the source code
being part of the \CONTEXT\ distribution, there is always a properly working,
more or less long term stable, snapshot.

The error reporting subsystem has been redone a little but is still fundamentally
the same. We don't really assume interactive usage but if someone uses it, it
might be noticed that it is not possible to backtrack or inject something. Of
course it is no big deal to implement all that in \LUA\ if needed. It removes a
system dependency and makes for a bit cleaner code.

There are new primitives too as well as some extensions to existing primitive
functionality. These are described in following chapters but there might be
hidden treasures in the binary. If you locate them, don't automatically assume
them to stay, some might be part of experiments!

\stopchapter

\stopcomponent

