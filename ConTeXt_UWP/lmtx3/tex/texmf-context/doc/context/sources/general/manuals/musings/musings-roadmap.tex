% language=uk

% \showfontkerns

\startcomponent musings-roadmap

\environment musings-style

\startchapter[title={\METATEX, a roadmap}]

% \startlines \setupalign[flushright]
% Hans Hagen
% Hasselt NL
% September 2018
% \stoplines

\startsection[title={Introduction}]

Here I will shortly wrap up the state of \LUATEX\ and \CONTEXT\ in fall 2018. I
made the first draft of this article as preparation for the \CONTEXT\ meeting
where we also discussed the future. I updated the text afterwards to match the
decisions made there. It's also a personal summary of thoughts and discussions
with team members about where to move next.

\stopsection

\startsection[title={The state of affairs}]

After a dozen years the development of \LUATEX\ has reached a state where adding
more functionality and|/|or opening up more of the internals makes not much
sense. Apart from fixes and maybe some minor extensions, version 1.10 is what you
get. Users can do enough in \LUA\ and there is not much to gain in convenience
and performance. Of course some of the code can and will be cleaned up, as we
still see the effects of going from \PASCAL\ to \CWEB\ to \CCODE. In the process
consistency is on the radar so we might occasionally add a helper. But we also
don't want to move too far away from the original code, which is for instance why
we keep names, keys and other properties found in original \TEX, which in turn
leads to some inconsistencies with extensions added over time. We have to accept
that.

Because \LUATEX\ development is closely related to \CONTEXT\ development,
especially \MKIV, we've also reached the moment that we can get rid of some older
code and assume the latest \LUATEX\ to be used. Because we do so much in \LUA\
the question is always to what extent the benefits outweigh the drawbacks. Just
in case you wonder why we use \LUA\ extensively, the main reason is that it is
easier and more efficient to manage data in this language and modern typesetting
needs much data. It also permits us to extend regular \TEX\ functionality. But,
one should not overrate the impact: we still let \TEX\ do what \TEX\ is best at!

Performance is quite important. It doesn't make sense to create a powerful
typesetting system where processing a page takes a second. We have discussed
performance before since one of the complaints about \LUATEX\ is that it is slow.
A simple, basic test is this:

\starttyping
\starttext
    \dorecurse{1000}{\input tufte \par}
\stoptext
\stoptyping

This involves 1000 times loading a file (and reporting that on the console, which
can influence runtime), typesetting paragraphs, splitting of a page and of course
loading fonts and saving to the \PDF\ file. When I run this on a modest machine,
I get these (relative) timings for the (about) 225 pages:

\starttabulate[|l|c|c|c|c|]
\BC \TEX\ engine used  \BC \PDFTEX \BC \LUATEX \BC \LUAJITTEX \BC \XETEX \NC \NR
\BC runtime in seconds \NC 2.0     \NC  3.9    \NC 3.0        \NC 8.4    \NC \NR
\stoptabulate

Now, as expected the 8 bit \PDFTEX\ is the winner here but \LUATEX\ is not doing
that bad. I don't know why \XETEX\ is so much slower, maybe because its 64 bit
binary is less optimal. I once noticed that a 64 bit \PDFTEX\ performed worse on
such a test than \LUATEX, for which I always use 64 bit binaries.

If you consider that often much more is done than in this example, you can take
my word that \LUATEX\ quickly outpaces \PDFTEX\ on more complex tasks. In that
sense it is now our benchmark. It must be said that the \MKIV\ code is probably a
bit more efficient than the \MKII\ code but that doesn't matter much in this
simple test because hardly any macro magic happens here; it mostly tests basic
font processing, paragraph building and page construction. I don't think that I
can squeeze out more pages per second, at least not without users telling me
where they encounter bottlenecks that don't result from their style coding. It's
no problem to write inefficient macros (or styles) so normally a user should
first carefully check her|/|his own work. Using a more modern \CPU\ with proper
caching and an \SSD\ helps too.

So, to summarize, we can say that with version 1.10 \LUATEX\ is sort of finished.
Our mission is now to make \LUATEX\ robust and stable. Things can be added and
improved, but these are small and mostly consistency related.

\stopsection

\startsection[title={More in \LUA}]

Till now I always managed to add functionality to \CONTEXT\ without hampering
performance too much. Of course the biggest challenge is always in handling fonts
and common features like color because that all happens in \LUA. So, the question
is, what if we delegate more of the core functionality to \LUA ? I will discuss a
few options because the \CONTEXT\ developers and users need to agree on the path
to follow. One question there is, are the possible performance hits (which can be
an inconvenience) compensated by better and easier typesetting.

Fonts, colors, special typesetting features like spaced kerning, protrusion,
expansion, but also dropped caps, line numbering, marginal notes, tables,
structure related things, floats and spacing are not open for much discussion.
All the things that happen in \LUA\ combined with macros is there and will stay.
But how about hyphenation, paragraph building and page building? And how about a
leaner and meaner, future safe engine?

Hyphenation is handled in the \TEX\ core. But in \CONTEXT\ already for years one
can also use a \LUA\ based variant. There is room for extensions and improvements
there. Interesting is that performance is more or less the same, so this is an
area where we might switch to the \LUA\ method eventually. It compares to fonts,
where node mode is more or less the standard and base mode the old way.

Building the paragraphs in \LUA\ is also available in \MKIV, although it needs an
update. Again performance is not that bad, so when we add features not possible
(or hard to do) in regular \TEX, it might actually pay of to default to the
par builder written in \LUA.

The page builder is also doable in \LUA\ but so far I only played a bit with a
\LUA\ based variant. I might pick up that thread. However, when we would switch to
\LUA\ there, it might have a bit of a penalty, unless we combine it with some
other mechanisms which is not entirely trivial, as it would mean a diversion from
the way \TEX\ does it normally.

How about math? We could at some point do math rendering in \LUA\ but because the
core mechanism is the standard, it doesn't really makes much sense. It would also
touch the soul of \TEX. But, I might give it a try, just for fun, so that I can
play with it a bit. It's typically something for cold and rainy days with some
music in the background.

We already use \LUA\ in the frontend: locating and reading files in \TEX,
\XML, \LUA\ and whatever input format. Normalization and manipulation is all
active and available. The backend is also depending on \LUA, like support for
special \PDF\ features and exporting to \XML . The engine still handles the page
stream conversion, font inclusion and object management.

The inclusion of images is also handled by the engine, although in \CONTEXT\ we
can delegate \PDF\ inclusion to \LUA. Interesting is that this has no performance
hit.

With some juggling the page stream conversion can also be done in \LUA, and I
might move that code into the \CONTEXT\ distribution. Here we do have a
performance hit: about one second more runtime on the 14 seconds needed for the
300 page \LUATEX\ manual and just over more than half a second on a 11 second
\LUAJITTEX\ run. The manual has lots of tables, verbatim, indices and uses color
as well as a more than average number of fonts and much time is spent in \LUA. So
there is a price to pay there. I tried to speed that up but there is not much to
gain there.

So, say that we default to \LUA\ based hyphenation, which enables some new
functionality, \LUA\ based par building, which permits some heuristics for corner
cases, and \LUA\ based page building, which might result in more control over
tricky cases. A total performance hit of some 5\% is probably acceptable,
especially because by that time I might have replaced my laptop and won't notice
the degrade. This still fits in the normal progress and doesn't really demand a
roadmap or wider acceptance. And of course we would still use the same strategies
as implemented in traditional \TEX\ as default anyway.

\stopsection

\startsection[title={A more drastic move}]

More fundamental is the question whether we delegate more backend activity to
\LUA\ code. If we decide to handle the page stream in \LUA, then the next
question is, why not also delegate object management and font inclusion to
\LUA. Now, keep in mind that this is all very \CONTEXT\ specific! Already for
more than a decade we delegate a lot to \LUA, and also we have a rather tight
control over this core functionality. This would mean that \CONTEXT\ doesn't
really need the backend code in the engine. \footnote {For generic packages like
TikZ we (can) provide some primitive emulators, which is rather trivial to
implement.}

That situation is actually not unique. For instance, already for a while we don't
need the \LUATEX\ font loader either, as loading the \OPENTYPE\ files is done in
\LUA. So, we could also get rid of the font loader code. Currently some code is
shared with the font inclusion in the backend but that can be isolated.

You can see a \TEX\ engine as being made from several parts, but the core really
concerns only two processes: reading, storing and expanding macros on the one
hand, and converting a stream of characters into lines, paragraphs, pages etc.
Fonts are mostly an abstraction: they are visible in so called glyph nodes as
font identifier (a number) and character code (also a number) properties. The
result, nowadays being \PDF, is also an abstraction: at some point the engine
converts the to be shipped out box in \PDF\ instructions, and in our case,
relatively simple ones. The backend registers which characters and fonts are used
and also includes the right resources. But, the backend is not part of the core
as such! It has been introduced in \PDFTEX\ and is a so called extension.

So, what does that all mean for a future version of \CONTEXT\ and \LUATEX ? It
means that we can decide to follow up with a \CONTEXT\ that does more in \LUA,
which means not hard coded in a binary, on the one hand, but that we can also
decide to strip the engine from non|-|core code. But, given that \LUATEX\ is also
used in other macro packages, this would mean a different engine. We cannot say
that \LUATEX\ is stable when we also experiment with core components.

We've seen folks picking up experimental versions assuming that it is a precursor
to official code. So, in order to move on we need to avoid confusion: we need to
use another name. Choosing a name is always tricky but as Taco already registered
the \METATEX\ domain, and because in the \CONTEXT\ distribution you will find
references to \METATEX, we will use that name for the future engine. Adding \LUA\
to that name makes sense but then the name would become too long.

The main difference between \METATEX\ and \LUATEX\ would be that the former has
no file lookup library, no hardcoded font loader, and no backend generator (but
possibly some helpers, and these need time to evolve). We're basically back where
\TEX\ started but instead of coding these extensions in \PASCAL\ or \CCODE\ we
use \LUA. We're also kind of back to when we first started experimenting with
\LUATEX\ in \CONTEXT\ where test, write and rewrite were going in parallel. But,
as said, we cannot impose that on a wide audience.

If we go for such a lean and mean follow up, then we can also do a more drastic
cleanup of obsolete code in \CONTEXT\ (dating from \ETEX, \PDFTEX, \ALEPH, etc.).
We then are sort of back to where it all started: we go back to the basics. This
might mean dropping some primitives (one can define them as dummy). Of course we
could generalize some of the \CONTEXT\ code to provide the kicked out
functionality but would that pay of? Probably not.

Just for the record: replacing the handling of macros, registers, grouping, etc.\
to \LUA\ is not really an option as the performance hit would make a large system
like \CONTEXT\ sort of unusable: it's no option and not even considered (although
I must admit that I have some experimental \LUA\ based \TEX\ parser code around).

It is quite likely that building \METATEX\ from source for the moment will be an
option to the build script. But we can also decide to simplify that process,
which is possible because we only need one binary. But in general we can assume
that one can generate \METATEX\ and \LUATEX\ from the same source. A first step
probably is a further isolation of the backend code. The fontloader and file
handling code already can be made optional.

Given that we only need one binary (it being \LUATEX\ or \METATEX) and nowadays
only use \OPENTYPE\ fonts, one can even start thinking of a mini distribution,
possibly with a zipped resource tree, something we experimented with in the early
days of \LUATEX.

Another though I have been playing with is a better separation between low level
and high level \CONTEXT\ commands, and whether the low level layer should be more
generic in nature (so that one can run specific packages on top of it instead of
the whole of \CONTEXT) but that might not be worth the trouble.

\stopsection

\startsection[title={Interlude}]

If we look at the future, it's good to also look at the past. Opening up \TEX\
the way we did has many advantages but also potential drawbacks. It works quite
well in \CONTEXT\ because we ship an integrated package. I don't think that there
are many users who kick in their own callbacks. It is possible but completely up
to the user to make sure things work out well. Performance hits, interference,
crashes: those who interfere with the internals can sort that out themselves. I'm
not sure how well that works out in other macro packages but it is a time bomb if
users start doing that. Of course the documented interfaces to use \LUA\ in
\CONTEXT\ are supported. So far I think we're not yet bitten in the tail. We keep
this aspect out of the discussion.

Another important aspect is stability of the engine. Sometimes we get suggestions
for changes or patches that works for a specific case but for sure will have side
effects on \CONTEXT. Just as we don't test \LATEX\ side effects, \LATEX\ users
don't check \CONTEXT. And we're not even talking of users who expect their code
to keep working. A tight control over the source is important but cannot be we
will not be around for ever. This means that at some point \LUATEX\ should not be
changed any more, even when we observe side effects we want to get rid of,
because these side effects can be in use. This is another argument for a stripped
down engine. The less there is to mess with, the less the mess.

\stopsection

\startsection[title={Audience}]

So how about \CONTEXT\ itself? Of course we can make it better. We can add more
examples and more documentation. We can try to improve support. The main question
for us (as developers) is who actually is our audience. From the mails coming to
the \CONTEXT\ support list it looks like a rather diverse group of users.

At \TEX\ meetings there are often discussions about promoting \TEX. I can agree
on the fact that even for simple documents it makes a lot of sense to use \TEX,
but who will take the first hurdles? How many people really produce a lot of
documents? And how many need \TEX\ after maybe a short period of (enforced) usage
at the university?

It's not trivial to recognize the possibilities and power of the
\LUATEX|-|\CONTEXT\ combination. We never got any serious requests for support
from large organizations. In fact, we do use this combination in a few projects
for educational publishers, but there it's actually the authors and editors doing
the work. It's seldom company policy to use tools that efficiently automate
typesetting. I dare to say that publishers are not really an audience at all:
they normally delegate the task. They might accept \TEX\ documents but let them
rekey or adapt far|-|far|-|away and as cheap as possible. Thinking of it, the
main reason for Don Knuth for writing \TEX\ in the first place was the ability to
control the look and feel and quality. It were developments at typesetters and
publishers that triggered development of \TEX . It was user demand. And the
success of \TEX\ was largely due to the unique personality and competence of the
author.

System integrators qualify as audience but I fear that \TEX\ is not considered
hip and modern. It doesn't seem to matter if you can demonstrate that it can do a
wonderful job efficiently and relatively cheap. Also the fact that an
installation can be very stable on the long run is of no importance. Maybe that
audience (market place) is all about \quotation {The more we have to program and
update regularly, the merrier.}. Marketing \TEX\ is difficult.

Those who render multiple products, maintain manuals, have to render many
documents automatically qualify as audience. But often company policies,
preferred suppliers, so called standard tools etc.\ are used as argument against
\TEX. It's a missed opportunity.

One needs a certain mindset to recognize the potential and the question is, how
do we reach that audience. Drawing a roadmap for that is not easy but worth
discussing. We're open for suggestions.

% \footnote {It's kind of interesting that recently the \TEX\ User Group announced
% its presence on Facebook and Twitter. Apart from wondering how that gets updated,
% one can also wonder how many potential (or even current) users go there, given
% that these platforms are subjected to rise and fall. I'm on neither of them and
% don't plan to. Kids (our future users) that I know already said goodbye to them.
% We'll see how that works out.}

\stopsection

\startsection[title={Conclusion}]

At the \CONTEXT\ user meeting those present agreed that moving forward this way
makes sense. This means that we will explore a lean and mean \METATEX\ alongside
\LUATEX. There is no rush and it's all volunteer work so we will take our time
for this. It boils down to some reshuffling of code so that we can remove the
built|-|in font loader, file handling, and probably also \SYNCTEX\ because we can
emulate that. Then the backend with its font inclusion code will be cleaned up a
bit (we even discussed only supporting modern wide fonts). It's no big deal to
adapt \CONTEXT\ to this (so it can and will support both \LUATEX\ and \METATEX).
Eventually the backend might go away but now we're talking years ahead. By then
we can also explore the option to make \METATEX\ start out as a \LUA\ function
call (the main control loop) and become reentrant. There will probably not be
many changes to the opened up \TEX\ kernel, but we might extend the \METAPOST\
part a bit (some of that was discussed at the meeting) especially because it is a
nice tool to visualize big data.

As with \LUATEX\ development we will go in small steps so that we keep a working
system. Of course \LUATEX\ is always there as stable fallback. The experiments
will mostly happen in the experimental branch and binaries will be generated
using the compile farm on the \CONTEXT\ garden, just as happens now. This also
limits testing and exploring to the \CONTEXT\ community so that there are no side
effects for mainstream \LUATEX\ usage.

Nowadays, instead if roadmaps, we tend to use navigational gadgets that adapt
themselves to the situation. On the road by car this can mean a detour and when
walking around it can be going to suggested points of interest. During the
excursion at the meeting, we noticed that after the drivers (navigators)
synchronized their gadget with Jano, the routes that were followed differed a
bit. We saw cars in front of going a different direction and cars behind us
arriving from a different direction. So, even when we talk about roadmaps, our
route can be adapted to the situation.

Now here is something to think about. If you look at the \TEX\ community you will
notice that it's an aging community. User groups seem to loose members, although
the \CONTEXT\ group is currently still growing. Fortunately we see a new
generation taking interest and the \CONTEXT\ users are a pleasant mix and it
makes me stay around. I see it as an \quote {old timers} responsibility to have
\TEX\ and its environment in a healthy state by the time I retire from it
(although I have no plans in that direction). In parallel to the upcoming
development I think we will also see a change in \TEX\ use and usage. This aspect
was also discussed at the meeting and for sure will get a follow up on the
mailing lists and future meetings. It might as well influence the decisions we
make the upcoming years. So far \TEX\ has never failed us in it's flexibility and
capacity to adapt, so let's end on that positive note.

\stopsection

\stopchapter

\stopcomponent
