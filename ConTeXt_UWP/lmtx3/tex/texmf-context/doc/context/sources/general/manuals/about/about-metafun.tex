% language=uk

\startcomponent about-metafun

\environment about-environment

\startchapter[title={\LUA\ in \METAPOST}]

% Hans Hagen, PRAGMA ADE, April 2014

\startsection[title=Introduction]

Already for some years I have been wondering how it would be if we could escape
to \LUA\ inside \METAPOST, or in practice, in \MPLIB\ in \LUATEX. The idea is
simple: embed \LUA\ code in a \METAPOST\ file that gets run as soon as it's seen.
In case you wonder why \LUA\ code makes sense, imagine generating graphics using
external data. The capabilities of \LUA\ to deal with that is more flexible and
advanced than in \METAPOST. Of course we could generate a \METAPOST\ definition
of a graphic from data but it often makes more sense to do the reverse. I finally
found time and reason to look into this and in the following sections I will
describe how it's done.

\stopsection

\startsection[title=The basics]

The approach is comparable to \LUATEX's \type {\directlua}. That primitive can be
used to execute \LUA\ code and in combination with \type {tex.print} we can pipe
strings back into the \TEX\ input stream. A complication is that we have to be
able to operate under different so called catcode regimes: the meaning of
characters can differ per regime. We also have to deal with line endings in
special ways as they relate to paragraphs and such. In \METAPOST\ we don't have
that complication so getting back input into the \METAPOST\ input, we can do so
with simple strings. For that a mechanism similar to \type {scantokens} can be
used. That way we can return anything (including nothing) as long as \METAPOST\
can interpret it and as long as it fulfils the expectations.

\starttyping
numeric n ; n := scantokens("123.456") ;
\stoptyping

A script is run as follows:

\starttyping
numeric n ; n := runscript("return '123.456'") ;
\stoptyping

This primitive doesn't have the word \type {lua} in its name so in principle any
wrapper around the library can use it as a hook. In the case of \LUATEX\ the
script language is of course \LUA. At the \METAPOST\ end we only expect a string.
How that string is constructed is completely up to the \LUA\ script. In fact, the
user is completely free to implement the runner any way she or he wants, like:

\starttyping
local function scriptrunner(code)
    local f = loadstring(code)
    if f then
        return tostring(f())
    else
        return ""
    end
end
\stoptyping

This is hooked into an instance as follows:

\starttyping
local m = mplib.new {
    ...
    run_script = scriptrunner,
    ...
}
\stoptyping

Now, beware, this is not the \CONTEXT\ way. We provide print functions and other
helpers, which we will explain in the next section.

\stopsection

\startsection[title=Helpers]

After I got this feature up and running I played a bit with possible interfaces
at the \CONTEXT\ (read: \METAFUN) end and ended up with a bit more advanced runner
where no return value is used. The runner is wrapped in the \type {lua} macro.

\startbuffer
numeric n ; n := lua("mp.print(12.34567)") ;
draw textext(n) xsized 4cm withcolor maincolor ;
\stopbuffer

\typebuffer

This renders as:

\startlinecorrection[blank]
\processMPbuffer
\stoplinecorrection

In case you wonder how efficient calling \LUA\ is, don't worry: it's fast enough,
especially if you consider suboptimal \LUA\ code and the fact that we switch
between machineries.

\startbuffer
draw image (
    lua("statistics.starttiming()") ;
    for i=1 upto 10000 : draw
        lua("mp.pair(math.random(-200,200),math.random(-50,50))") ;
    endfor ;
    setbounds currentpicture to fullsquare xyscaled (400,100) ;
    lua("statistics.stoptiming()") ;
    draw textext(lua("mp.print(statistics.elapsedtime())"))
        ysized 50 ;
) withcolor maincolor withpen pencircle scaled 1 ;
\stopbuffer

\typebuffer

Here the line:

\starttyping
draw lua("mp.pair(math.random(-200,200),math.random(-50,50))") ;
\stoptyping

effectively becomes (for instance):

\starttyping
draw scantokens "(25,40)" ;
\stoptyping

which in turn becomes:

\starttyping
draw scantokens (25,40) ;
\stoptyping

The same happens with this:

\starttyping
draw textext(lua("mp.print(statistics.elapsedtime())")) ...
\stoptyping

This becomes for instance:

\starttyping
draw textext(scantokens "1.23") ...
\stoptyping

and therefore:

\starttyping
draw textext(1.23) ...
\stoptyping

We can use \type {mp.print} here because the \type {textext} macro can deal with
numbers. The following also works:

\starttyping
draw textext(lua("mp.quoted(statistics.elapsedtime())")) ...
\stoptyping

Now we get (in \METAPOST\ speak):

\starttyping
draw textext(scantokens (ditto & "1.23" & ditto) ...
\stoptyping

Here \type {ditto} represents the double quotes that mark a string. Of course,
because we pass the strings directly to \type {scantokens}, there are no outer
quotes at all, but this is how it can be simulated. In the end we have:

\starttyping
draw textext("1.23") ...
\stoptyping

What print variant you use, \type {mp.print} or \type {mp.quoted}, depends on
what the expected code is: an assignment to a numeric can best be a number or an
expression resulting in a number.

This graphic becomes:

\startlinecorrection[blank]
\processMPbuffer
\stoplinecorrection

The runtime on my current machine is some 0.25 seconds without and 0.12 seconds
with caching. But to be honest, speed is not really a concern here as the amount
of complex \METAPOST\ graphics can be neglected compared to extensive node list
manipulation. Generating the graphic with \LUAJITTEX\ takes 15\% less time.
\footnote {Processing a small 8 page document like this takes about one second,
which includes loading a bunch of fonts.}

\startbuffer
numeric n ; n := lua("mp.print(1) mp.print('+') mp.print(2)") ;
draw textext(n) xsized 1cm withcolor maincolor ;
\stopbuffer

The three print command accumulate their arguments:

\typebuffer

As expected we get:

\startlinecorrection[blank]
\processMPbuffer
\stoplinecorrection

\startbuffer
numeric n ; n := lua("mp.print(1,'+',2)") ;
draw textext(n) xsized 1cm withcolor maincolor ;
\stopbuffer

Equally valid is:

\typebuffer

This gives the same result:

\startlinecorrection[blank]
\processMPbuffer
\stoplinecorrection

Of course all kind of action can happen between the prints. It is also legal to
have nothing returned as could be seen in the 10.000 dot example: there the timer
related code returns nothing, so effectively we have \type {scantokens("")}.
Another helper is \type {mp.quoted}, as in:

\startbuffer
draw
    textext(lua("mp.quoted('@0.3f'," & decimal n & ")"))
    withcolor maincolor ;
\stopbuffer

\typebuffer

This typesets \processMPbuffer. Note the \type {@}. When no percent character is
found in the format specifier, we assume that an \type {@} is used instead.

\startbuffer
\startluacode
table.save("demo-data.lua",
    {
        { 1, 2 }, { 2, 4 }, { 3, 3 }, { 4, 2 },
        { 5, 2 }, { 6, 3 }, { 7, 4 }, { 8, 1 },
    }
)
\stopluacode
\stopbuffer

But, the real benefit of embedded \LUA\ is when we deal with data that is stored
at the \LUA\ end. First we define a small dataset:

\typebuffer

\getbuffer

There are several ways to deal with this table. I will show clumsy as well as
better looking ways.

\startbuffer
lua("MP = { } MP.data = table.load('demo-data.lua')") ;
numeric n ;
lua("mp.print('n := ',\#MP.data)") ;
for i=1 upto n :
    drawdot
        lua("mp.pair(MP.data[" & decimal i & "])") scaled cm
        withpen pencircle scaled 2mm
        withcolor maincolor ;
endfor ;
\stopbuffer

\typebuffer

Here we load a \LUA\ table and assign the size to a \METAPOST\ numeric. Next we
loop over the table entries and draw the coordinates.

\startlinecorrection[blank]
\processMPbuffer
\stoplinecorrection

We will stepwise improve this code. In the previous examples we omitted wrapper
code but here we show it:

\startbuffer
\startluacode
    MP.data = table.load('demo-data.lua')
    function MP.n()
        mp.print(#MP.data)
    end
    function MP.dot(i)
        mp.pair(MP.data[i])
    end
\stopluacode

\startMPcode
    numeric n ; n := lua("MP.n()") ;
    for i=1 upto n :
        drawdot
            lua("MP.dot(" & decimal i & ")") scaled cm
            withpen pencircle scaled 2mm
            withcolor maincolor ;
    endfor ;
\stopMPcode
\stopbuffer

\typebuffer

So, we create a few helpers in the \type {MP} table. This table is predefined so
normally you don't need to define it. You may however decide to wipe it clean.

\startlinecorrection[blank]
\getbuffer
\stoplinecorrection

You can decide to hide the data:

\startbuffer
\startluacode
    local data = { }
    function MP.load(name)
        data = table.load(name)
    end
    function MP.n()
        mp.print(#data)
    end
    function MP.dot(i)
        mp.pair(data[i])
    end
\stopluacode
\stopbuffer

\typebuffer \getbuffer

It is possible to use less \LUA, for instance in:

\startbuffer
\startluacode
    local data = { }
    function MP.loaded(name)
        data = table.load(name)
        mp.print(#data)
    end
    function MP.dot(i)
        mp.pair(data[i])
    end
\stopluacode

\startMPcode
    for i=1 upto lua("MP.loaded('demo-data.lua')") :
        drawdot
            lua("MP.dot(",i,")") scaled cm
            withpen pencircle scaled 4mm
            withcolor maincolor ;
    endfor ;
\stopMPcode
\stopbuffer

\typebuffer

Here we also omit the \type {decimal} because the \type {lua} macro is clever
enough to recognize it as a number.

\startlinecorrection[blank]
\getbuffer
\stoplinecorrection

By using some \METAPOST\ magic we can even go a step further in readability:

\startbuffer
\startMPcode{doublefun}
    lua.MP.load("demo-data.lua") ;

    for i=1 upto lua.MP.n() :
        drawdot
            lua.MP.dot(i) scaled cm
            withpen pencircle scaled 4mm
            withcolor maincolor ;
    endfor ;

    for i=1 upto MP.n() :
        drawdot
            MP.dot(i) scaled cm
            withpen pencircle scaled 2mm
            withcolor white ;
    endfor ;
\stopMPcode
\stopbuffer

\typebuffer

Here we demonstrate that it also works well in \type {double} mode, which makes
much sense when processing data from other sources. Note how we omit the
type {lua} prefix: the \type {MP} macro will deal with that.

\startlinecorrection[blank]
\getbuffer
\stoplinecorrection

So in the end we can simplify the code that we started with to:

\starttyping
\startMPcode{doublefun}
    for i=1 upto MP.loaded("demo-data.lua") :
        drawdot
            MP.dot(i) scaled cm
            withpen pencircle scaled 2mm
            withcolor maincolor ;
    endfor ;
\stopMPcode
\stoptyping

\stopsection

\startsection[title=Access to variables]

The question with such mechanisms is always: how far should we go. Although
\METAPOST\ is a macro language, it has properties of procedural languages. It also
has more introspective features at the user end. For instance, one can loop over
the resulting picture and manipulate it. This means that we don't need full
access to \METAPOST\ internals. However, it makes sense to provide access to
basic variables: \type {numeric}, \type {string}, and \type {boolean}.

\startbuffer
draw textext(lua("mp.quoted('@0.15f',mp.get.numeric('pi')-math.pi)"))
    ysized 1cm
    withcolor maincolor ;
\stopbuffer

\typebuffer

In double mode you will get zero printed but in scaled mode we definitely get a
different results:

\startlinecorrection[blank]
\processMPbuffer
\stoplinecorrection

\startbuffer
boolean b ; b := true ;
draw textext(lua("mp.quoted(mp.get.boolean('b') and 'yes' or 'no')"))
    ysized 1cm
    withcolor maincolor ;
\stopbuffer

In the next example we use \type {mp.quoted} to make sure that indeed we pass a
string. The \type {textext} macro can deal with numbers, but an unquoted \type
{yes} or \type {no} is asking for problems.

\typebuffer

Especially when more text is involved it makes sense to predefine a helper in
the \type {MP} namespace, if only because \METAPOST\ (currently) doesn't like
newlines in the middle of a string, so a \type {lua} call has to be on one line.

\startlinecorrection[blank]
\processMPbuffer
\stoplinecorrection

Here is an example where \LUA\ does something that would be close to impossible,
especially if more complex text is involved.

% \enabletrackers[metapost.lua]

\startbuffer
string s ; s := "ΤΕΧ" ; % "τεχ"
draw textext(lua("mp.quoted(characters.lower(mp.get.string('s')))"))
    ysized 1cm
    withcolor maincolor ;
\stopbuffer

\typebuffer

As you can see here, the whole repertoire of helper functions can be used in
a \METAFUN\ definition.

\startlinecorrection[blank]
\processMPbuffer
\stoplinecorrection

\stopsection

\startsection[title=The library]

In \CONTEXT\ we have a dedicated runner, but for the record we mention the
low level constructor:

\starttyping
local m = mplib.new {
    ...
    script_runner = function(s) return loadstring(s)() end,
    script_error  = function(s) print(s) end,
    ...,
}
\stoptyping

An instance (in this case \type {m}) has a few extra methods. Instead you can use
the helpers in the library.

\starttabulate[|l|l|]
\HL
\NC \type {m:get_numeric(name)}       \NC returns a numeric (double) \NC \NR
\NC \type {m:get_boolean(name)}       \NC returns a boolean (\type {true} or \type {false}) \NC \NR
\NC \type {m:get_string (name)}       \NC returns a string \NC \NR
\HL
\NC \type {mplib.get_numeric(m,name)} \NC returns a numeric (double) \NC \NR
\NC \type {mplib.get_boolean(m,name)} \NC returns a boolean (\type {true} or \type {false}) \NC \NR
\NC \type {mplib.get_string (m,name)} \NC returns a string \NC \NR
\HL
\stoptabulate

In \CONTEXT\ the instances are hidden and wrapped in high level macros, so there
you cannot use these commands.

\stopsection

\startsection[title=\CONTEXT\ helpers]

The \type {mp} namespace provides the following helpers:

\starttabulate[|l|l|]
\HL
\NC \type {print(...)}                \NC returns one or more values \NC \NR
\NC \type {pair(x,y)}
    \type {pair(t)}                   \NC returns a proper pair \NC \NR
\NC \type {triplet(x,y,z)}
    \type {triplet(t)}                \NC returns an \RGB\ color \NC \NR
\NC \type {quadruple(w,x,y,z)}
    \type {quadruple(t)}              \NC returns an \CMYK\ color \NC \NR
\NC \type {format(fmt,...)}           \NC returns a formatted string \NC \NR
\NC \type {quoted(fmt,...)}
    \type {quoted(s)}                 \NC returns a (formatted) quoted string  \NC \NR
\NC \type {path(t[,connect][,close])} \NC returns a connected (closed) path \NC \NR
\HL
\stoptabulate

The \type {mp.get} namespace provides the following helpers:

\starttabulate[|l|l|]
\NC \type {numeric(name)}      \NC gets a numeric from \METAPOST \NC \NR
\NC \type {boolean(name)}      \NC gets a boolean from \METAPOST \NC \NR
\NC \type {string(name)}       \NC gets a string from \METAPOST \NC \NR
\HL
\stoptabulate

\stopsection

\startsection[title=Paths]

% {\em This section will move to the metafun manual.} \blank

In the meantime we got several questions on the \CONTEXT\ mailing list about turning
coordinates into paths. Now imagine that we have this dataset:

\startbuffer[dataset]
10 20 20 20 -- sample 1
30 40 40 60
50 10

10 10 20 30 % sample 2
30 50 40 50
50 20

10 20 20 10 # sample 3
30 40 40 20
50 10
\stopbuffer

\typebuffer[dataset]

In this case I have put the data in a buffer, so that it can be shown
here, as well as used in a demo. Look how we can add comments. The
following code converts this into a table with three subtables.

\startbuffer
\startluacode
  MP.myset = mp.dataset(buffers.getcontent("dataset"))
\stopluacode
\stopbuffer

\typebuffer \getbuffer

We use the \type {MP} (user) namespace to store the table. Next we turn
these subtables into paths:

\startbuffer
\startMPcode
  for i=1 upto lua("mp.print(mp.n(MP.myset))") :
    draw
      lua("mp.path(MP.myset[" & decimal i & "])")
      xysized (HSize,10ExHeight)
      withpen pencircle scaled .25ExHeight
      withcolor basiccolors[i]/2 ;
  endfor ;
\stopMPcode
\stopbuffer

\typebuffer

This gives:

\startlinecorrection[blank] \getbuffer \stoplinecorrection

Instead we can fill the path, in which case we will also need to close it. The
\type {true} argument deals with that:

\startbuffer
\startMPcode
  for i=1 upto lua("mp.print(mp.n(MP.myset))") :
    path p ; p :=
      lua("mp.path(MP.myset[" & decimal i & "],true)")
      xysized (HSize,10ExHeight) ;
    fill p
      withcolor basiccolors[i]/2
      withtransparency (1,.5) ;
  endfor ;
\stopMPcode
\stopbuffer

\typebuffer

We get:

\startlinecorrection[blank] \getbuffer \stoplinecorrection

\startbuffer
\startMPcode
  for i=1 upto lua("mp.print(mp.n(MP.myset))") :
    path p ; p :=
      lua("mp.path(MP.myset[" & decimal i & "])")
      xysized (HSize,10ExHeight) ;
    p :=
      (xpart llcorner boundingbox p,0) --
      p --
      (xpart lrcorner boundingbox p,0) --
      cycle ;
    fill p
      withcolor basiccolors[i]/2
      withtransparency (1,.25) ;
  endfor ;
\stopMPcode
\stopbuffer

The following makes more sense:

\typebuffer

So this gives:

\startlinecorrection[blank] \getbuffer \stoplinecorrection

This (area) fill is so common, that we have a helper for it:

\startbuffer
\startMPcode
  for i=1 upto lua("mp.size(MP.myset)") :
    fill area
      lua("mp.path(MP.myset[" & decimal i & "])")
      xysized (HSize,5ExHeight)
      withcolor basiccolors[i]/2
      withtransparency (2,.25) ;
  endfor ;
\stopMPcode
\stopbuffer

\typebuffer

So this gives:

\startlinecorrection[blank] \getbuffer \stoplinecorrection

This snippet of \METAPOST\ code still looks kind of horrible, so how can we make
it look better? Here is an attempt. First we define a bit more \LUA:

\startbuffer
\startluacode
local data = mp.dataset(buffers.getcontent("dataset"))

MP.dataset = {
  Line = function(n) mp.path(data[n]) end,
  Size = function()  mp.size(data)    end,
}
\stopluacode
\stopbuffer

\typebuffer \getbuffer

\startbuffer
\startMPcode
  for i=1 upto lua.MP.dataset.Size() :
    path p ; p :=
      lua.MP.dataset.Line(i)
      xysized (HSize,20ExHeight) ;
    draw
      p
      withpen pencircle scaled .25ExHeight
      withcolor basiccolors[i]/2 ;
    drawpoints
      p
      withpen pencircle scaled ExHeight
      withcolor .5white ;
  endfor ;
\stopMPcode
\stopbuffer

We can now make the \METAPOST\ look more natural. Of course, this is possible
because in \METAFUN\ the \type {lua} macro does some extra work.

\typebuffer

As expected, we get the desired result:

\startlinecorrection[blank] \getbuffer \stoplinecorrection

Once we start making things look nicer and more convenient, we quickly end up
with helpers like those in the next example. First we save some demo data in
files:

\startbuffer
\startluacode
  io.savedata("foo.tmp","10 20 20 20 30 40 40 60 50 10")
  io.savedata("bar.tmp","10 10 20 30 30 50 40 50 50 20")
\stopluacode
\stopbuffer

\typebuffer \getbuffer

We load the data in datasets:

\startbuffer
\startMPcode
  lua.mp.datasets("load","foo","foo.tmp") ;
  lua.mp.datasets("load","bar","bar.tmp") ;
  fill area
    lua.mp.datasets("foo","line")
    xysized (HSize/2-EmWidth,10ExHeight)
    withpen pencircle scaled .25ExHeight
    withcolor green/2 ;
  fill area
    lua.mp.datasets("bar","line")
    xysized (HSize/2-EmWidth,10ExHeight)
    shifted (HSize/2+EmWidth,0)
    withpen pencircle scaled .25ExHeight
    withcolor red/2 ;
\stopMPcode
\stopbuffer

\typebuffer

Because the datasets are stored by name, we can use them without worrying about
them being forgotten:

\startlinecorrection[blank] \getbuffer \stoplinecorrection

If no tag is given, the filename (without suffix) is used as a tag, so the
following is valid:

\starttyping
\startMPcode
  lua.mp.datasets("load","foo.tmp") ;
  lua.mp.datasets("load","bar.tmp") ;
\stopMPcode
\stoptyping

The following methods are defined for a dataset:

\starttabulate[|l|pl|]
\HL
\NC \type {method} \NC usage \NC \NR
\HL
\NC \type {Size}   \NC the number of subsets in a dataset \NC \NR
\NC \type {Line}   \NC the joined pairs in a dataset making a non|-|closed path \NC \NR
\NC \type {Data}   \NC the table containing the data (in subsets, so there is always at least one subset) \NC \NR
\HL
\stoptabulate

{\em Due to limitations in \METAPOST\ suffix handling the methods start with an
uppercase character.}

\stopsection

\startsection[title=Remark]

The features described here are currently still experimental but the interface
will not change. There might be a few more accessors and for sure more \LUA\
helpers will be provided. As usual I need some time to play with it before I make
up my mind. It is also possible to optimize the \METAPOST||\LUA\ script call a
bit, but I might do that later.

When we played with this interface we ran into problems with loop variables
and macro arguments. These are internally kind of anonymous. Take this:

\starttyping
for i=1 upto 100 : draw(i,i) endfor ;
\stoptyping

The \type {i} is not really a variable with name \type {i} but becomes an object
(capsule) when the condition is scanned, and a reference to that object when the
body is scanned. The body of the for loop gets expanded for each step, but at that
time there is no longer a variable \type {i}. The same is true for variables in:

\starttyping
def foo(expr x, y, delta) = draw (x+delta,y+delta) enddef ;
\stoptyping

We are still trying to get this right with the \LUA\ interface. Interesting is
that when we were exploring this, we ran into quite some cases where we could
make \METAPOST\ abort due some memory or stack overflow. Some are just bugs in
the new code (due to the new number model) while others come with the design of
the system: border cases that never seem to happen in interactive use while the
library use assumes no interaction in case of errors.

In \CONTEXT\ there are more features and helpers than shown here but these are
discussed in the \METAFUN\ manual.

\stopsection

\stopchapter

\stopcomponent

% \startMPcode{doublefun}
%     numeric n ; n := 123.456 ;
%     lua("print('>>>>>>>>>>>> number',mp.get.number('n'))") ;
%     lua("print('>>>>>>>>>>>> number',mp.get.boolean('n'))") ;
%     lua("print('>>>>>>>>>>>> number',mp.get.string('n'))") ;
%     boolean b ; b := true ;
%     lua("print('>>>>>>>>>>>> boolean',mp.get.number('b'))") ;
%     lua("print('>>>>>>>>>>>> boolean',mp.get.boolean('b'))") ;
%     lua("print('>>>>>>>>>>>> boolean',mp.get.string('b'))") ;
%     string s ; s := "TEST" ;
%     lua("print('>>>>>>>>>>>> string',mp.get.number('s'))") ;
%     lua("print('>>>>>>>>>>>> string',mp.get.boolean('s'))") ;
%     lua("print('>>>>>>>>>>>> string',mp.get.string('s'))") ;
% \stopMPcode

