\environment leaflet-common

\startdocument[graphic=1]

\startbuffer[1]
    There are several ways to deploy \CONTEXT. Most common is to let it render a
    document, in which case you install it on a system and use an editor to input
    your document and from a console or by clicking some key trigger a run. The
    input can be structured using \TEX\ macros but it can also be \XML, or some
    other format that gets converted to \CONTEXT\ commands before processing.
\stopbuffer

\startbuffer[2]
    Instead you can also use \CONTEXT\ as a more hidden application, for instance
    in a web service or rendering component in a larger application. In that case
    the end user is not really aware that \TEX\ is being used.
\stopbuffer

\startbuffer[3]
    No matter how you use \CONTEXT, you need to install it first. You can for
    instance use \TEX live or another distribution to pick up \CONTEXT, but you
    can also install it using the archive (snapshot), in which case you also need
    to pick up the engine (for instance \LUATEX) and a basic set of fonts. In
    order to make installation easy we provide a so called standalone
    distribution that has all you need.
\stopbuffer

\startbuffer[4]

    The standalone \CONTEXT\ distribution has the following characteristics:

    \startitemize
        \startitem
            The installation is self contained. Apart from resources like fonts,
            the \TEX\ macros, \LUA\ code and \METAPOST\ helpers are provided in
            one package.
        \stopitem
        \startitem
            There is only one binary involved: \LUAMETATEX. The source code of
            this program is integral part of the \CONTEXT\ distribution (per end
            2019). A user should be able to compile the program if needed. There
            is no dependency on additional libraries other than those that make
            up the operating system.
        \stopitem
        \startitem
            The core system is able to typeset documents in an efficient way. The
            memory footprint is decent and performance acceptable, also on on low
            power devices and virtual machines. We try not to provide bloatware.
        \stopitem
        \startitem
            The official user interface is stable and the implementation of core
            components will not change fundamentally. When something can be
            improved it will be. One can use a snapshot for long term stability.
        \stopitem
        \startitem
            Support is provided by means of a mailing list, a wiki,
            documentation, meetings, etc. If needed you can consult (or hire)
            support. There are enough experienced users out there to get you
            going.
        \stopitem
    \stopitemize
\stopbuffer

\startbuffer[5]
    The first version of \CONTEXT, now tagged \MKII, has been around since 1995
    and (still) runs on top of \PDFTEX. The development of its successor \MKIV\
    started in 2005 as part of the \LUATEX\ development and still carries on. The
    most recent incarnation is \LMTX, which is \MKIV\ but tuned for \LUAMETATEX,
    the lean and mean successor of \LUATEX.
\stopbuffer

% we could use a collector

\startpagemakeup
    \offinterlineskip
    \vskip24pt
    \hbox to \hsize \bgroup
        \hss
        \startframed[align=normal,width=.7\textwidth,frame=off]
            \getbuffer[1]
        \stopframed
        \hskip1cm
    \egroup
    \vfill
    \hbox to \hsize \bgroup
        \hskip1cm
        \startframed[align=normal,width=.7\textwidth,frame=off]
            \getbuffer[2]
        \stopframed
    \egroup
    \vfill
    \hbox to \hsize \bgroup
        \hss
        \startframed[align=normal,width=.7\textwidth,frame=off]
            \getbuffer[3]
        \stopframed
        \hskip1cm
    \egroup
    \vfill
    \hbox to \hsize \bgroup
        \hss
        \setupitemize[after=]
        \startframed[align=normal,width=\dimexpr\textwidth-2cm\relax,frame=off]
            \getbuffer[4]
        \stopframed
        \hskip1cm
    \egroup
    \vfill
    \hbox to \hsize \bgroup
        \hss
        \startframed[align=normal,width=.7\textwidth,frame=off]
            \getbuffer[5]
        \stopframed
        \hskip1cm
    \egroup
    \vfill
    \vfill
    \vfill
    \hbox to \hsize \bgroup
        \hskip1cm
        \startframed[align=normal,width=.7\textwidth,frame=off]
            \bfd \CONTEXT\ \LMTX \enspace \emdash \enspace factsheet
        \stopframed
        \hss
    \egroup
    \vskip12pt
\stoppagemakeup

\stopdocument
