% language=uk

\startcomponent languages-introduction

\environment languages-environment

\startchapter[title=Introduction][color=darkgray]

This document describes an important property of the \TEX\ typesetting system and
\CONTEXT\ in particular: the ability to deal with different languages at the same
time. With languages we refer to natural languages. So, we're not going to
discuss the \TEX\ language itself, not \METAPOST, nor \LUA.

The original application of \TEX\ was English that uses the Latin script. The
fonts that came with \TEX\ were suitable for that usage. When lines became too
long they could be hyphenated using so called hyphenation patterns. Due to the
implementation for many years there was a close relationship between fonts and
hyphenation. Although at some point many more languages and scripts were
supported, it was only when the \UNICODE\ aware variants showed up that
hyphenation and fonts were decoupled. This makes it much more easier to mix
languages that use different scripts. Although Greek, Cyrillic, Arabic, Chinese,
Japanese, Korean and other languages have been supported for a while using
(sometimes dirty) tricks, we now have cleaner implementations.

We can hyphenate words in all languages (and scripts) that have a need for it,
that is, split it at the end of a line and add a symbol before and|/|or after the
break. The way words are broken into parts is called hyphenation and so called
patterns are used to achieve that goal. The way these patterns are constructed
and applied was part of the research related to \TEX\ development. The method
used is also applied in other programs and is probably one of the few popular
ways to deal with hyphenation. There have been ideas about extensions that cover
the demands of certain languages but so far nothing better has shown up. In the
end \TEX\ does a pretty decent job and more advanced tricks don't necessarily
lead to better results.

Hyphenation is driven by a language number and that's about it. This means that
one cannot claim that \TEX\ in its raw form supports languages, other than that
it can hyphenate and use fonts that provide the glyphs. It's upto a macro package
to wrap this into a mechanism that provides the user an interface. So, when we
speak about language support, hyphenation is only one aspect. Labels, like the
\type {figure} in {\em figure~1.2} need to adapt to the main document language.
When dates are shown they can be language specific. Scientific units and math
function names can also be subjected to translation. Registers and other lists
have to be sorted according to specific rules. Spacing dan differ per language.

In this manual we will cover some of functionality in \CONTEXT\ \MKIV\ that
relates to languages (and scripts). This manual is a compliment to other manuals,
articles and documentation. Here we mostly focus on the language aspects. Some of
the content (or maybe most) might looks alien and complex to you. This is because
one purpose of this manual is to provide a place to wrap up some aspects of
\CONTEXT. If you're not interested in that, just stick to the more general
manuals that also cover language aspects.

\startnotabene
    This document is still under construction. The functionality discussed here
    will stay and more might show up. Of course there are errors, and they're all
    mine. The text is not checked for spelling errors. Feel free to let me know
    what should get added.
\stopnotabene

\startlines
Hans Hagen
PRAGMA ADE, Hasselt NL
2013 \emdash\ 2016
\stoplines

\stopchapter

\stopcomponent
