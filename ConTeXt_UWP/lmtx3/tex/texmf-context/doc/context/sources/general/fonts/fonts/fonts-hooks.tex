% language=uk

\startcomponent fonts-hooks

\environment fonts-environment

\startchapter[title=Hooks][color=darkcyan]

\startsection[title=Introduction]

One of the virtues of \TEX\ is its flexibility. Because we cannot predict what
users want to mess around with, much of the underlying code has hooks. And because
it's not too hard to add functionality that will break things we will not advocate
all of it. Of course you can study the code and figure out what can be done and
there is no problem with that. It's just that you shouldn't expect much support.

In this chapter we collect some of these hooks. If you run into interesting ones
that are worth mentioning, you can always ask us to add description here.

\stopsection

\startsection[title=Safe hooks]

\startsubsection[title=Trimming fonts]

Because we store font related information in \LUA\ tables there can be situations
where the resources used outgrow memory. An example of such a font is \type
{lastresort} that basically defined the whole \UNICODE\ range. The font is
actually not that large as it uses similar placeholders for glyphs in a range,
but it has rather verbose (redundant) names. As we normally don't need these, you
can decide to strip them away.

\starttyping
\startluacode
    fonts.handlers.otf.readers.registerextender {
        name   = "remove names from lastresort",
        action = function(fontdata)
            if fontdata.metadata.fullname == "LastResort" then
                for k, v in next, fontdata.descriptions do
                    v.name = nil
                end
            end
        end
    }
\stopluacode

\definedfont[LastResort][lastresort*default sa 1]
\stoptyping

This will result in a much smaller font, one that has less change to crash the
engine due to lack of memory. Extenders like this are applied once the font has
been loaded but before it gets saved.

\stopsubsection

\stopsection

\startsection[title=Loading]

\startsubsection[title=Introduction]

We basically have to deal with three font formats that can easily be recognized
by the suffix of the files involved: \type {tfm} and \type {vf} files that
describe 8 bit fonts, traditionally bitmap fonts, but as they carry only metric
information, any 8 bit font can be described. Then there are \type {afm} files
that contain metrics related to \TYPEONE\ fonts (stored in \type {pfb} files).
Although such fonts could contain more than 256 shapes, the implementation was
limited to 8 bits too. By converting \type {afm} files to \type {tfm} files,
traditional \TEX\ can deal with \TYPEONE\ given that the backend can include them
in the final result.

In this section we will discuss some aspects of the \OPENTYPE\ font reader. As
\TEX\ only deals with metrics (in the frontend) we need to parse them, filter
information from it and pass the metrics to \TEX. In addition, we can use all
kind of extra information to manipulate the so called node list but in the end
\TEX\ is only interested in font id's (that point to a font resource) and glyph
indexes.

To overcome the 256 limitation of \TYPEONE\ fonts, in \CONTEXT\ we moved away
from \type {tfm} files (we can of course still deal with them) and turn \type
{afm} files into so called wide fonts. Basically we turn them in a more rich
format that looks similar to the internal \OPENTYPE\ format we use. We will not
go into much detail about that because \TYPEONE\ is kind of obsolete and being
replaced by \OPENTYPE, but we will of course support the old formats simply
because we have all these fonts around.

Already early in the development of \LUATEX\ a font loader library was created
that can turn an \OPENTYPE\ (but also a \TYPEONE) font into a \LUA\ table. This
library is derived from \FONTFORGE\ which makes it possible to look into a font
using that editor and at the same time get a similar view on the font in \LUA,
which is quite handy. However, at some point in \CONTEXT\ we wanted to play with
outlines in \METAPOST\ and for that purpose an \OPENTYPE\ reader was written in
\LUA\ that could extract the data. Because \TYPEONE\ fonts already were done in
\LUA\ it was a logical step to also do \OPENTYPE\ in \LUA\ so now we use an
alternative loader that doesn't depend in the \FONTFORGE\ library. This not only
gives more flexibility but also makes it possible to avoid some conversions
needed to provide the \CONTEXT\ font handler with the needed information in an
efficient way.

\stopsubsection

\startsubsection[title=Loading \OPENTYPE\ fonts]

As with most binary media formats today an \OPENTYPE\ font file is a linked list
of records. The top level structure is called table. There are two flavours of
\OPENTYPE\ where the main difference is in the way the shapes are defined: they
can be \TRUETYPE\ outlines using quadratric bezier curves or cff files using
cubic bezier curves. The last variant is the same as \POSTSCRIPT\ \TYPEONE\
fonts. Simplified, a quadratic curve defines the shape in points with a control
point in between, while a quadratic one also has points but each with two control
points (as in \METAPOST).

An \OPENTYPE\ font can be large: there can be upto 65536 glyphs and lots of extra
properties and features. In order to save space the data is rather packed using
different numeric data types. Of course one can wonder if size really matters now
that most bandwidth is taken by audio, video and pictures but we have to live
with it.

The definition of \OPENTYPE\ can be found on the \MICROSOFT\ website:
\hyphenatedurl {https://www.microsoft.com/typography/otspec}. Most tables then
could make sense for us are mentioned in the following list:

\starttabulate[|Bl|l|l|]
\NC required    \NC cmap \NC character to glyph mapping \NC \NR
\NC             \NC head \NC font header \NC \NR
\NC             \NC hhea \NC horizontal header \NC \NR
\NC             \NC hmtx \NC horizontal metrics \NC \NR
\NC             \NC maxp \NC maximum profile \NC \NR
\NC             \NC name \NC naming table \NC \NR
\NC             \NC os/2 \NC os/2 and windows specific metrics \NC \NR
\NC             \NC post \NC postScript information \NC \NR
\NC truetype    \NC glyf \NC glyph data \NC \NR
\NC             \NC loca \NC index to location \NC \NR
\NC postscript  \NC cff  \NC compact font format \NC \NR
\NC             \NC vorg \NC vertical origin \NC \NR
\NC typographic \NC base \NC baseline data \NC \NR
\NC             \NC gdef \NC glyph definition data \NC \NR
\NC             \NC gpos \NC glyph positioning data \NC \NR
\NC             \NC gsub \NC glyph substitution data \NC \NR
\NC             \NC jstf \NC justification data \NC \NR
\NC             \NC math \NC math layout data \NC \NR
\NC extras      \NC kern \NC kerning \NC \NR
\NC             \NC ltsh \NC linear threshold data \NC \NR
\NC             \NC vhea \NC vertical metrics header \NC \NR
\NC             \NC vmtx \NC vertical metrics \NC \NR
\NC             \NC colr \NC color table \NC \NR
\NC             \NC cpal \NC color palette table \NC \NR
\stoptabulate

When we read these tables it depends on what we want to do with the result how
much we will really read. For instance when we only want to identify a font and
get some basic information we don't need to read all tables and certainly don't
need to read them completely. If we want to have the outlines we need to read the
\type {glyf} or \type {cff} table. If we also want to boundingbox of \POSTSCRIPT\
shapes we even need to process the shapes so that we know the dimensions of the
result. There is no need to summarize the format here in detail because you can
find it on the \MICROSOFT\ site. Here I only cover some aspects that influence
the way \TEX\ can use the fonts.

One of the main differences between the readers is that the \FONTFORGE\ reader
has a lot of (recovery) heuristics for bad fonts. Nowadays most fonts are quite
okay, and in \CONTEXT\ we prefer to just reject bad ones. In the process of
loading the built|-|in loader gives each glyph a name (it makes them up for
variants needed for features). It also tries to figure out some font properties,
like the weight. If does a pretty good job on that but it is also hard to repair
at the \LUA\ end when it makes a bad guess. The \LUA\ variants stays closer to
the specification, but delegates more to the final user, which is good because we
need and want that level of control as controls is what \TEX\ is about. It also
made it possible to support for instance colored fonts without too much effort.

So what data needs to be collected? If we look at what we get eventually the list
of glyphs is the bulk. For each glyph we collect some metric information. For
instance we fetch the (advance) width of the glyph but also the boundingbox,
which gives us the the height and depth.

In the font file the list of glyphs starts at zero and runs up tot the total
number of glyphs. The index in this table is used in for instance the tables that
define the font features, for instance kerning between glyphs, or multiple glyphs
that are turned into ligatures. Each glyph gets a name. That can be a meaningful
one but also a rather dumb one, for instance the index number.

Eventually (at least in \CONTEXT) we don't order by glyph index but by \UNICODE.
The font file contains information about the mapping from index to \UNICODE. In
principle other encodings are possible but we stick to \UNICODE. But, because
many glyphs can refer to one \UNICODE\ slot, for instance a regular shape as well
as a smallcaps or oldstyle variant. These extra glyphs we let end up in the
private \UNICODE\ areas. This also means that with each glyph in the final table
there is also a field that has the \UNICODE. Because we order by \UNICODE\ we
also need to store the index. An example from a Latin Modern font is:

\starttyping
[97] = {
    boundingbox = { 34, -10, 474, 446 },
    index       = 28,
    name        = "a",
    unicode     = 97,
    width       = 490,
}
\stoptyping

Another example is the following. Here we end up in private space:

\starttyping
[983059] = {
    boundingbox = { 30, -10, 734, 446 },
    index       = 19,
    name        = "oe.dup",
    unicode     = 339,
    width       = 762,
}
\stoptyping

Yet another entry is:

\starttyping
[306] = {
   boundingbox = { 28, -22, 790, 683 },
   index       = 357,
   name        = "I_J",
   unicode     = { 73, 74 },
   width       = 839,
  },
\stoptyping

Here you see two \UNICODE\ numbers. That kind of information is deduced from the
name of the glyph, using knowledge on how such names are supposed to be
constructed, or, when that is not possible, from ligature information in the
fonts.

It makes no sense to discuss the whole font table in detail, if only because most users
will never (need to) see it. But if your curious you can have a look at the fonts
in the cache tree, in the \CONTEXT\ distribution from the \CONTEXT\ garden this is

\starttyping
.../tex/texmf-cache/luatex-cache/context/<somehash>/fonts/otl
\stoptyping

There can be three kind of files there, with suffixes \type {tma}, \type {tmc}
and \type {tmb}. The first one is the table as converted from the binary font
file. The second and third variants are just bytecode compilations of this file
(for \LUATEX\ and|/|or \LUAJITTEX). The bytecode variants are smaller but more
important, they load a bit faster. On my disk the largest \type{tma} file is just
below 10 MByte (an extensive \CJK\ font) but normally they are in the few hundred
KByte range (some are real small), with the bytecode files of course being
relatively small to their original.

However, there is a bit of cheating here. If we run the command:

\starttyping
mtxrun --script font --convert lmroman10-regular.otf
\stoptyping

A \LUA\ file is generated: \type {lmroman10-regular.lua}. This file is much larger
than the \type {tma} file in the cache:

\starttabulate[|T|T|]
\NC 643.924 \NC lmroman10-regular.lua \NC 0.029 \NR
\NC 209.950 \NC lmroman10-regular.tma \NC 0.010 \NR
\NC 121.541 \NC lmroman10-regular.tmb \NC \NR
\NC 134.564 \NC lmroman10-regular.tmc \NC 0.003 \NR
\stoptabulate

The reason for this is the following. Most information is stored in tables.
Especially tables that describe font features can be the same all over the place.
This is why we pack the table in a more compact format before saving it in the
cache, and unpack it after loading. The effects on loading are neglectable but
and it has the benefit that it saves a lot of memory. By looking at such numbers
one should be careful with conclusions, but (assuming proper garbage collection)
we see a memory footprint of the \type {lua} file of 2836 Kbyte, while the
unpacked variant takes 704 Kbyte. You can imagine what happens with large \CJK\
fonts. Loading the (larger unpacked) \type {lua} file currently costs me 0.029
seconds, while loading and unpacking the \type {tma} file takes 0.010 seconds and
the bytecode variant \type {tmc} 0.003 seconds.

\stopsubsection

\startsubsection[title=Loading \TYPEONE\ fonts]

When we started with \CONTEXT\ \MKIV\ (which is shortly after we started with
\LUATEX) the only \TFM\ files that were loaded, were those to make virtual
\UNICODE\ math fonts, awaiting real \OPENTYPE\ math fonts. Math fonts are kind
of special with respect to metrics and such.

For \TYPEONE\ text fonts we didn't use the \TFM\ files but went for parsing \AFM\
files. That way we could use all the glyphs provided by fonts and not be limited
to 256 slots. So, effectively we made them \UNICODE\ and similar to \OPENTYPE. Of
course the only features were ligatures, kerns and some special ones like \TEX\
ligatures and replacements. With the old loader code, we always made them base
mode fonts, which means that processing was delegated to \TEX. In the new loader
we implement ligatures and kerns as node mode features, which means that we can
use those fonts in base mode as well as node mode. The last options therefore
permits to add or adapt features to \TYPEONE\ fonts as well.

In the next sections we will focus on \OPENTYPE\ but as the \TYPEONE\ fonts are
organized in a similar way, some of it also applies to this older type. The most
important to keep in mind is that we only have \type {liga}, \type {kern} and a
few \CONTEXT\ specific features.

\stopsubsection

\stopsection

\startsection[title=The tables]

\startsubsection[title=Structure]

Getting a font read for \TEX\ happens in stages. The original \OPENTYPE\ file is
read only once. At that moment the shapes are described in the \type
{descriptions} subtable while by the time that we pass the information to \TEX\
they are in \type {characters}. The reason is that we go from dimensions in font
units to dimensions in scaled points. We start with the following table:

\ctxlua{context.tocontext(fonts.tables.data.original,"original_table")}

The table passed \TEX\ is constructed from this one and looks like:

\ctxlua{context.tocontext(fonts.tables.data.scaled,"scaled_table")}

There might be a few more (often obscure) fields for special purposes. The
characters subtable conforms to what \TEX\ expects, while the descriptions stays
closer to \OPENTYPE. The \type {kerns} and \type {ligatures} subtables are there
for base mode and are not present in \type {node} mode. The \type {commands} and
\type {fonts} subtables relate to virtual fonts.

\startitemize[packed]
\startitem
    Start with the (already) loaded \OPENTYPE\ table.
\stopitem
\startitem
    Copy relevant information from \type {descriptions} to \type {characters} etc.
\stopitem
\startitem
    Construct \type {properties} and \type {parameters} tables.
\stopitem
\startitem
    Apply additional manipulators, for instance extend the \type {characters}
    table, with expansion and protrusion.
\stopitem
\startitem
    Scale the \type {characters}, \type {properties} and \type {parameters}.
\stopitem
\startitem
    Apply additional manipulators.
\stopitem
\startitem
    Pass the table to \TEX, but keep it around for later access.
\stopitem
\stopitemize

One of the things you need to be aware of is that all references to glyphs are
\UNICODE\ slots, either natural ones (representing a character) or a private one
(representing an alternative representation). In \OPENTYPE\ features are defined
in terms of glyph indices but we prefer \UNICODE\ because that is easier to deal
with when we run over the node list. Before font processing the character field
in a glyph node is a \UNICODE\ slot and afterwards it's still a \UNICODE\ but
when it's a private one it can always be resolved to a non private slot of
sequence of slots. Of course that could also be done with indices but it's just
more natural this way.

Another thing to note is that in the descriptions we're still working with font
units ranging from $-1000$ to $+1000$, $-2048$ to $+2048$ or similar ranges. At
the \TEX\ end we need scaled points which are much larger numbers.

The question is: how often do users need to access the raw data in a font? After
a decade of \MKIV\ and \LUATEX\ hardly any user has requested such access,
probably because when needed easier interfaces were provided. Also, in the
\CONTEXT\ distrubution there are some examples of manipulations that can be
copied and adapted to personal use. There's also a danger is messing with the
fonts (similar messing with the node lists): you never know how it interferes
with other (maybe future) features.

If you still want to do it, best is probably to start with saving the
to|-|be|-|passed|-|to|-|\TEX\ table in a file and have a look at it. The most
prominent subtable is the \type {characters} table and messing a bit with
dimensions is rather harmless. You could add characters, for instance virtual
ones, which again is harmless unless you use invalid commands. You probably want
to stay away from the resources subtable, if only because some of its subtables
are shared and therefore adapting them can have side effects. The top level \type
{shared} and \type {unscaled} subtable are off limits as is the \type
{specification}.

You can save a font by consulting one of the hashes but for a specific font
you need to know its id. You can do this by using low level accessors but better
is to use the helpers made for this, because they prevent saving redundant
data.

% \starttyping
% \startluacode
% local nullfont    = fonts.hashes.identifiers[false]
% local currentfont = fonts.hashes.identifiers[true]
%
% local id, tfmdata = fonts.definers.define {
%     name = "dejavusansmono*default",
%     size = tex.sp("6pt")
% }
%
% table.save("temp-nullfont.lua",   nullfont)
% table.save("temp-currentfont.lua",currentfont)
% table.save("temp-definedfont.lua",tfmdata)
% table.save("temp-definedfont.lua",fonts.hashes.identifiers[id])
% \stopluacode
% \stoptyping

\starttyping
\startluacode
fonts.tables.save  {
    filename = "temp-font-scaled.lua",
    fontname = "dejavusansmono*default",
    method   = "original",
}
\stopluacode
\stoptyping

At the \TEX\ end you can use:

\starttyping
\savefont
  [name=dejavusansmono*default,
   file=temp-o.lua,
   method=original]
\savefont
  [name=dejavusansmono*default,
   file=temp-s.lua,
   method=scaled]
\stoptyping

When no \type {name} is given, the current font is used and when no \type {file}
is given a filename is made up. The default \type {method} is \type {scaled}. The
saved name is reported.

\stopsubsection

\startsubsection[title=Plug-ins]

There are several places where you can hook in code: before scaling
(initalizers), after scaling (manipulators) and while processing (processors).
Only the first two are meant for tweaks.

\starttyping
local do_something = {
    name        = "something",
    description = "doing something",
    initializers = {
     -- position = 1,
        base     = function(tfmdata,value,features) ... end,
        node     = function(tfmdata,value,features) ... end,
    },
    manipulators = {
     -- position = 1,
        base     = function(tfmdata,feature,value) ... end,
        node     = function(tfmdata,feature,value) ... end,
    },
    processors = {
     -- position = 1,
        base     = function(tfmdata,font,attr) ... end,
        node     = function(tfmdata,font,attr) ... end,
    }
}

fonts.constructors.features.register.otf(so_something)
fonts.constructors.features.register.afm(so_something)
\stoptyping

A \type {initializer} is applied just before the font gets scaled. This means
that the characterm properties and parameters are unscaled! Initializers can for
instance be used to add extra features to fonts. You can provide an \type
{position} key with a number to force a place in the list of initializers but of
course you can never be sure of interference.

A \type {manipulator} is applied when the font is scaled but before it gets
passed to \TEX. It's a good place to tweak dimensions. Here you can also probide
a \type {position}.

The processors are applied when the node list gets processed, hence the \type
{font} and optional \type {attr} arguments. The action is only applied to the
specified font (id) and when an attribute gets passed, this is tested for a
value. When an attribute is used, an unset attribute on the node will skip the
action.

If adapting characters and their properties is your main objetive, then there is a
better plugin mechanism using sequencers. We illustrate this with a fake example:

\starttyping
\startluacode

function document.b_copying(tfmdata)
    logs.report("fonts","before copying: %s",tfmdata.properties.filename)
end
function document.a_copying(tfmdata)
    logs.report("fonts","after copying: %s",tfmdata.properties.filename)
end

function document.b_math(tfmdata)
    logs.report("fonts","before math: %s",tfmdata.properties.filename)
end
function document.a_math(tfmdata)
    logs.report("fonts","after math: %s",tfmdata.properties.filename)
end

utilities.sequencers.appendaction(
    "beforecopyingcharacters",
    "before",
    "document.a_copying"
)

utilities.sequencers.appendaction(
    "aftercopyingcharacters",
    "after",
    "document.b_copying"
)

utilities.sequencers.appendaction(
    "mathparameters",
    "before",
    "document.b_math"
)

utilities.sequencers.appendaction(
    "mathparameters",
    "after",
    "document.a_math"
)
\stopluacode
\stoptyping

When we call the next command:

\starttyping
\definedfont[MathRoman at 3pt]
\stoptyping

we get this reported:

\starttyping
fonts > before math: ...../public/dejavu/texgyredejavu-math.otf
fonts > after math: ...../public/dejavu/texgyredejavu-math.otf
fonts > after copying: ...../public/dejavu/texgyredejavu-math.otf
fonts > before copying: ...../public/dejavu/texgyredejavu-math.otf
\stoptyping

In between \type {before} and \type {after} we have \type {system} which is
reserved for \CONTEXT\ actions. These actions are executed in the scaler
function. The function get two tables passed: the original data as well as the
target. If you ever need these hooks, you can probably best run an \type
{inspect} on these arguments to see what you're dealing with.

Fonts get reused when possible and for that a hash is calculated depending on the
enabled features and size. If for some reason you want to adapt that hash you can
use postprocessors. When the \type {tfmdata} table has a subtable \type
{postprocessors}, then the actions in that subtable will be applied. When an
action returns a string, the string will be combined with the hash. You can set
(o rextend) the postprocessors table using the previopusly mentioned commands.
However, in \CONTEXT\ you can best stay away from this as it might interfere. This
mechanism is mostly provided for generic use.

\stopsubsection

\stopsection

\startsection[title=Goodies]

The font goodies are already discussed as an official mechanism to extend or enhance
fonts with additional features. There are quite some goodies defined and for sure more will
show up. Here is the full repertoire:

\ctxlua{context.tocontext(fonts.tables.data.goodies,"goodie_table")}

Of course you will never use all the options at the same time. The best place to
look for examples are the \type {lfg} files in the \CONTEXT\ distribution.
\footnote {At some point we might decide to also support goodies in the generic
version.}

\stopsection

\startsection[title=Extra characters]

When \TEX\ loads a font it gets stored in an internal format. Although \LUATEX\
can still load \TFM\ files, normally we will load font data using \LUA\ and then
pass it to \TEX. When that is done the font is basically frozen. After all, once
you entered text and that text has been processed to a sequence of glyphs, the
backend has to be sure that it can include the right data in the result. What
matters there is:

\startitemize[packed]
\startitem the width of a glyph \stopitem
\startitem the index of a glyph in the font \stopitem
\startitem properties of a font, like its name \stopitem
\startitem all kind manipulations don't with a virtual glyph\stopitem
\stopitemize

So, changing an already used glyph is not done. But, adding a new one should not
be a big deal. So, starting with \LUATEX\ 1.0.5 we can add characters (that
become glyphs) to a font after it got passed to \TEX.

Of course there are some limitations to this. For instance, when \OPENTYPE\
features are needed, you also need to extend that bit and it's not that trivial.
But adding independent special characters is no problem. Also, you can decide to
replace an already processed glyph by another one newly created with the same
dimensions but a different rendering.

Here I'll give a few (simple) examples. First we define a helper that creates a
rule. After that we use all kind of \CONTEXT\ data structures and helpers but the
general setup is not that complicated.

\startbuffer
\startluacode
    function document.MakeGlyph(w)
        local v = 655360
        local w = w*v
        local h = v
        local d = v/3
        return {
            width  = w,
            height = h,
            depth  = d,
            commands = {
                { "down", d },
                { "rule", h + d, w }
            },
        }
    end
\stopluacode
\stopbuffer

\typebuffer \getbuffer

Of course, when one defines a font to be passed to \TEX\ it needs to conform to
the standard. The \LUATEX\ manual describes what parameters and other properties
have to be set. We cheat and copy the so called null font. We also create a fonts
sub table. In such a table, a reference to id zero will be resolved to the id of
the font.

After defining the font table, we pass it to \TEX\ with \type {font,define} watch
how we also pass an already reserved id. Then we add some characters and we also
replace character 122 which is no problem because, after all, we haven't used it
yet. We just use numbers so don't think we're doing \UNICODE\ here.

\startbuffer
\startluacode
    local fontdata = fonts.hashes.identifiers

    local id = font.nextid(true) -- true reserves the id
    local nf = table.copy(fontdata[0])

    local make = document.MakeGlyph
    local add  = font.addcharacters

    nf.name       = "foo"
    nf.type       = "virtual"
    nf.fonts      = { { id = 0 } }
    nf.characters = {
        [122] = make(1),
        [123] = make(2),
    }

    font.define(id,nf)

    fontdata[id] = nf

    local t = make(3)
    nf.characters[124] = t
    add(id, {
        fonts      = nf.fonts,
        characters = { [124] = t }
    })

    local t = make(4)
    nf.characters[125] = t
    add(id, {
        fonts      = nf.fonts,
        characters = { [125] = t }
    })

    local t = make(8)
    nf.characters[122] = t
    add(id, {
        fonts      = nf.fonts,
        characters = { [122] = t }
    })

    interfaces.setmacro("MyDemoFontId",id)
\stopluacode
\stopbuffer

\typebuffer \getbuffer

\startbuffer
\def\MyDemoFont{\setfontid\MyDemoFontId}
\stopbuffer

We also define a command to access this font:

\typebuffer \getbuffer

\startbuffer
{\MyDemoFont \type{[122=}\char122\type{]}}
{\MyDemoFont \type{[123=}\char123\type{]}}
{\MyDemoFont \type{[124=}\char124\type{]}}
{\MyDemoFont \type{[125=}\char125\type{]}}
\stopbuffer

and we test this font as follows:

\typebuffer

This gives:

\startlines \getbuffer \stoplines

Next we extend an existing font and demonstrate several methods for extending a
font. First we define a font that we will patch.

\startbuffer
\definefontfeature[myextension-one][default][myextension=1]

\definefont[MyDemoOne][Serif*myextension-one]
\stopbuffer

\typebuffer \getbuffer

\startbuffer
\startluacode
    local currentfont   = font.current()
    local fontdata      = fonts.hashes.identifiers[currentfont]
    local characters    = fontdata.characters
    local cfonts        = fontdata.fonts
    local addcharacters = font.addcharacters

    local make = document.MakeGlyph

    local function startextension()
        statistics.starttiming()
    end

    local function stopextension(n)
        context.NC() context.formatted.type("\\char%s",n)
        context.NC() context.char(n)
        context.NC() context("%0.3f",statistics.stoptiming())
        context.NC() context.NR()
    end

    context.starttabulate { "||||" }

    startextension()
        for i=1000,1999 do
            local t = make(3)
            characters[i] = t
            addcharacters(currentfont, {
                fonts      = cfonts,
                characters = { [i] = t }
            })
        end
    stopextension(1500)

    startextension()
        local t = {
            fonts      = cfonts,
            characters = characters
        }
        for i=2000,2999 do
            characters[i] = make(5)
            addcharacters(currentfont, t)
        end
    stopextension(2500)

    startextension()
        for i=3000,3999 do
            characters[i] = make(7)
        end
        addcharacters(currentfont, {
            fonts      = cfonts,
            characters = characters
        })
    stopextension(3500)

    startextension()
        local t = { }
        for i=4000,4999 do
            characters[i] = make(9)
            t[i] = characters[i]
        end
        addcharacters(currentfont, {
            fonts      = cfonts,
            characters = t
        })
    stopextension(4500)

    local addcharacters = fonts.constructors.addcharacters

    startextension()
        local t = { }
        for i=5000,5999 do
            t[i] = make(11)
        end
        addcharacters(currentfont, {
            fonts      = cfonts,
            characters = t
        })
    stopextension(5500)

    context.stoptabulate()
\stopluacode
\stopbuffer

Watch how we only pass the new characters. We also need to make sure that the
table at the \LUA\ end gets updated, because we might need the data later. You
can see that not all methods are equally efficient. The last extension uses a
helper that also makes sure that the main character table at the \LUA\ end gets
updated.

\typebuffer \start \MyDemoOne \getbuffer \stop

\startbuffer
\startluacode
    local addcharacters = fonts.constructors.addcharacters
    local currentfont   = font.current()
    local parameters    = fonts.hashes.parameters[currentfont]

    local m = metapost.simple
    local f = string.formatters
        ["draw fullsquare rotated %i scaled %b randomized 2bp withcolor %s"]

    local push = { "push" }
    local pop  = { "pop" }

    function make()
        local r = parameters.size
        local o = r/2
        local p1 = m("metafun",f( 0, r, "red"))
        local p2 = m("metafun",f(30, r, "green"))
        local p3 = m("metafun",f(60, r, "blue"))
        return {
            width  = o + r,
            height = 2*o,
            depth  = o,
            commands = {
                { "down", -o/2 }, { "right", o/2 + o },
                push, { "pdf", "origin", p1 }, pop,
                push, { "pdf", "origin", p2 }, pop,
                push, { "pdf", "origin", p3 }, pop,
            },
        }
    end

    local t = { }
    for i=6000,6010 do
        t[i] = make()
    end
    addcharacters(currentfont, {
        fonts      = cfonts,
        characters = t
    })
\stopluacode
\stopbuffer

In this example we use \METAPOST\ to generate a shape. There is some juggling
with dimensions and we need to shift the image in order to get a proper baseline.

\typebuffer \start \MyDemoOne \showglyphs \getbuffer \stop

These shapes show up as follows. We show the bounding box too:

\startbuffer
\scale [width=\textwidth] {%
    \char6000 \space
    \char6001 \space
    \char6002 \space
    \char6003
}
\stopbuffer

\typebuffer \startlinecorrection \MyDemoOne \showglyphs \getbuffer \stoplinecorrection

When defining virtual characters you need to keep in mind that there are limits to
how large a character can be. If you go too far \LUATEX\ will quit with a scale
related message.

In \CONTEXT\ there are a couple of mechanism that were implemented when \LUATEX\
came around that can be updated to use the tricks described here. I'm not sure if
I'll do that. After all, it works now too. The main benefit of the fact that we
can extend a font within reasonable bounds is that future mechanism can benefit
from it.

There is one thing to keep in mind. Say that we define a font like this:

\starttyping
\definefont[MyDemoOneX][Serif*myextension-one]
\stoptyping

Because we already defined a font with the same size, we automatically get the characters
\type {6000} upto \type {6003}. If we don't want this and want a fresh instance, we can do:

\starttyping
\definefontfeature[myextension-two][default][myextension=2]
\definefont[MyDemoOneX][Serif*myextension-two]
\stoptyping

or just:

\starttyping
\definefont[MyDemoOneX][Serif*default]
\stoptyping

Normally this kind of hackery is not arbitrary and part of a well designed set up
so one knows what one gets.

\stopsection

% - features
% - subfonts
% - outlines
% - math
% - hashes

\stopchapter

\stopcomponent
