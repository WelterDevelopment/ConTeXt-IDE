% language=uk

\startcomponent about-hz

\environment about-environment

\startchapter[title={Font expansion}]

\startsection[title=Introduction]

A lot in \LUATEX\ is not new. It started as a mix of \PDFTEX\ (which itself is
built on top of original \TEX\ and \ETEX) and the directional bits of \ALEPH\
(which is a variant of \OMEGA). Of course large portions have been changed in the
meantime, most noticeably the input encoding (\UNICODE), fonts with a more
generic fontloader and \LUA\ based processing, \UNICODE\ math and related font
rendering, and many subsystems can be overloaded or extended. But at the time I
write this (end of January 2013) the parbuilder still has the \PDFTEX\ font
expansion code.

This code is the result of a research project by \THANH. By selectively widening
shapes a better greyness of the paragraph can be achieved. This trick is inspired
by the work of Hermann Zapf and therefore, instead of expansion, we often talk of
{\em hz} optimization.

It started with (runtime) generated \METAFONT\ bitmap fonts and as a consequence
we ended up with many more font instances. However, when eventually bitmap
support was dropped and outlines became the norm, the implementation didn't
change much. Also some of the real work was delegated to the backend and as it
goes then: never change a working system if there's no reason.

When I played with the \LUA\ based par builder I quickly realized that this
implementation was far from efficient. It was already known that enabling it
slowed down par building and I saw that this was largely due to many redundant
calculations, generating auxiliary fonts, and the interaction between front- and
backend. And, as I seldom hesitate to reimplement something that can be done
better (one reason why \CONTEXT\ is never finished) I came to an alternative
implementation. That was 2010. What helped was that by that time Hartmut Henkel
already had made the backend part cleaner, in the sense that instead of including
multiple instances of the same font (but with different glyph widths) the base
font was transformed in|-|line. This made me realize that we could use just one
font in the frontend and pass the scale with the glyph node to the backend. And
so, an extra field was added to glyphs nodes in order to make experiments
possible.

More than two years later (January 2013) I finally took up this pet project and
figured out how to adapt the backend. Interestingly a few lines of extra code
we all that was needed. At the same time the frontend part became much simpler,
that is, in the \LUA\ parbuilder. But eventually it will be retrofitted into the
core engine, if only because that's much faster.

\stopsection

\startsection[title=The changes]

The most important changes are the following. Instead of multiple font instances,
only one is used. This way less memory is used, no extra font instances need to
be created (and those \OPENTYPE\ fonts can be large).

Because less calculations are needed the code looks less complex and more elegant.
Okay, the parbuilder code will never really look easy, if only because much more
is involved.

The glyph related factors are related to the emwidth. This makes not much sense
so in \CONTEXT\ we define them in fractions of the character width, map them onto
emwidths, and in the parbuilder need to go to glyph related widths again. If we can
get rid of these emwidths, we have less complex code.

Probably for reasons of efficiency an expanded font carries a definition that
tells how much stretch and shrink is permitted and how large the steps are. So,
for instance a font can be widened 5\% and narrowed 3\% in steps of 1\% which
gives at most 8 extra instances. There is no real reason why this should be a
font property and the parbuilder cannot deal with fonts with different steps
anyway, so it makes more sense to make it a property of the paragraph and treat
all fonts alike. In the \LUA\ based variant we can even have more granularity but
we leave that for now. In any case this will lift the limitation of mixed font
usage that is present in the original mechanism.

The front- and backend code with repect to expansion gets clearly separated. In
fact, the backend doesn't need to do any calculations other than applying the
factor that is carried with the glyph. This and previously mentioned simplifications
make the mechanism more efficient.

It is debatable if expansion needs to be applied to font kerns, as is the case in
the old mechanism. So, at least it should be an option. Removing this feature
would again made the code nicer. If we keep it, we should keep in mind that
expansion doesn't work well with complex fonts (say Arabic) but I will look into
this later. It might be feasible when using the \LUA\ based variant because then
we can use some of the information that is carried around with the related
mechanisms. Of course this then related to the \LUA\ based font builder.

\stopsection

\stopchapter

\stopcomponent

