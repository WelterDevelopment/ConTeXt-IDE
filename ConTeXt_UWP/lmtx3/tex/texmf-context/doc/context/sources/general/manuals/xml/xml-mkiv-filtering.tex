\environment xml-mkiv-style

\startcomponent xml-mkiv-filtering

\startchapter[title={Filtering content}]

\startsection[title={\TEX\ versus \LUA}]

It will not come as a surprise that we can access \XML\ files from \TEX\ as well
as from \LUA. In fact there are two methods to deal with \XML\ in \LUA. First
there are the low level \XML\ functions in the \type {xml} namespace. On top of
those functions there is a set of functions in the \type {lxml} namespace that
deals with \XML\ in a more \TEX ie way. Most of these have similar commands at
the \TEX\ end.

\startbuffer
\startxmlsetups first:demo:one
  \xmlfilter {#1} {artist/name[text()='Randy Newman']/..
    /albums/album[position()=3]/command(first:demo:two)}
\stopxmlsetups

\startxmlsetups first:demo:two
  \blank \start \tt
    \xmldisplayverbatim{#1}
  \stop \blank
\stopxmlsetups

\xmlprocessfile{demo}{music-collection.xml}{first:demo:one}
\stopbuffer

\typebuffer

This gives the following snippet of verbatim \XML\ code. The indentation is
conform the indentation in the whole \XML\ file. \footnote {The (probably
outdated) \XML\ file contains the collection stores on my slimserver instance.
You can use the \type {mtxrun --script flac} to generate such files.}

\doifmodeelse {atpragma} {
    \getbuffer
} {
    \typefile{xml-mkiv-01.xml}
}

An alternative written in \LUA\ looks as follows:

\startbuffer
\blank \start \tt \startluacode
  local m = lxml.load("mine","music-collection.xml") -- m == lxml.id("mine")
  local p = "artist/name[text()='Randy Newman']/../albums/album[position()=4]"
  local l = lxml.filter(m,p) -- returns a list (with one entry)
  lxml.displayverbatim(l[1])
\stopluacode \stop \blank
\stopbuffer

\typebuffer

This produces:

\doifmodeelse {atpragma} {
    \getbuffer
} {
    \typefile{xml-mkiv-02.xml}
}

You can use both methods mixed but in practice we will use the \TEX\ commands in
regular styles and the mixture in modules, for instance in those dealing with
\MATHML\ and cals tables. For complex matters you can write your own finalizers
(the last action to be taken in a match) in \LUA\ and use them at the \TEX\ end.

\stopsection

\startsection[title={a few details}]

In \CONTEXT\ setups are a rather common variant on macros (\TEX\ commands) but
with their own namespace. An example of a setup is:

\starttyping
\startsetup doc:print
  \setuppapersize[A4][A4]
\stopsetup

\startsetup doc:screen
  \setuppapersize[S6][S4]
\stopsetup
\stoptyping

Given the previous definitions, later on we can say something like:

\starttyping
\doifmodeelse {paper} {
  \setup[doc:print]
} {
  \setup[doc:screen]
}
\stoptyping

Another example is:

\starttyping
\startsetup[doc:header]
  \marking[chapter]
  \space
  --
  \space
  \pagenumber
\stopsetup
\stoptyping

in combination with:

\starttyping
\setupheadertexts[\setup{doc:header}]
\stoptyping

Here the advantage is that instead of ending up with an unreadable header
definitions, we use a nicely formatted setup. An important property of setups and
the reason why they were introduced long ago is that spaces and newlines are
ignored in the definition. This means that we don't have to worry about so called
spurious spaces but it also means that when we do want a space, we have to use
the \type {\space} command.

The only difference between setups and \XML\ setups is that the following ones
get an argument (\type {#1}) that reflects the current node in the \XML\ tree.

\stopsection

\startsection[title={CDATA}]

What to do with \type {CDATA}? There are a few methods at tle \LUA\ end for
dealing with it but here we just mention how you can influence the rendering.
There are four macros that play a role here:

\starttyping
\unexpanded\def\xmlcdataobeyedline {\obeyedline}
\unexpanded\def\xmlcdataobeyedspace{\strut\obeyedspace}
\unexpanded\def\xmlcdatabefore     {\begingroup\tt}
\unexpanded\def\xmlcdataafter      {\endgroup}
\stoptyping

Technically you can overload them but beware of side effects. Normally you won't
see much \type {CDATA} and whenever we do, it involves special data that needs
very special treatment anyway.

\stopsection

\startsection[title={Entities}]

As usual with any way of encoding documents you need escapes in order to encode
the characters that are used in tagging the content, embedding comments, escaping
special characters in strings (in programming languages), etc. In \XML\ this
means that in order characters like \type {<} you need an escape like \type
{&lt;} and in order then to encode an \type {&} you need \type {&amp;}.

In a typesetting workflow using a programming language like \TEX, another problem
shows up. There we have different special characters, like \type {$ $} for triggering
math, but also the backslash, braces etc. Even one such special character is already
enough to have yet another escaping mechanism at work.

Ideally a user should not worry about these issues but it helps figuring out issues
when you know what happens under the hood. Also it is good to know that in the
code there are several ways to deal with these issues. Take the following document:

\starttyping
<text>
    Here we have a bit of a &lt;&mess&gt;:

    # &#35;
    % &#37;
    \ &#92;
    { &#123;
    | &#124;
    } &#125;
    ~ &#126;
</text>
\stoptyping

When the file is read the \type {&lt;} entity will be replaced by \type {<} and
the \type {&gt;} by \type {>}. The numeric entities will be replaced by the
characters they refer to. The \type {&mess} is kind of special. We do preload
a huge list of more or less standardized entities but \type {mess} is not in
there. However, it is possible to have it defined in the document preamble, like:

\starttyping
<!DOCTYPE dummy SYSTEM "dummy.dtd" [
    <!ENTITY mess "what a mess" >
]>
\stoptyping

or even this:

\starttyping
<!DOCTYPE dummy SYSTEM "dummy.dtd" [
    <!ENTITY mess "<p>what a mess</p>" >
]>
\stoptyping

You can also define it in your document style using one of:

\startxmlcmd {\cmdbasicsetup{xmlsetentity}}
    replaces entity with name \cmdinternal {cd:name} by \cmdinternal {cd:text}
\stopxmlcmd

\startxmlcmd {\cmdbasicsetup{xmltexentity}}
    replaces entity with name \cmdinternal {cd:name} by \cmdinternal {cd:text}
    typeset under a \TEX\ regime
\stopxmlcmd

Such a definition will always have a higher priority than the one defined
in the document. Anyway, when the document is read in all entities are
resolved and those that need a special treatment because they map to some
text are stored in such a way that we can roundtrip them. As a consequence,
as soon as the content gets pushed into \TEX, we need not only to intercept
special characters but also have to make sure that the following works:

\starttyping
\xmltexentity {tex} {\TEX}
\stoptyping

Here the backslash starts a control sequence while in regular content a
backslash is just that: a backslash.

Special characters are really special when we have to move text around
in a \TEX\ ecosystem.

\starttyping
<text>
    <title>About #3</title>
</text>
\stoptyping

If we map and define title as follows:

\starttyping
\startxmlsetup xml:title
    \title{\xmlflush{#1}}
\stopxmlsetup
\stoptyping

normally something \type {\xmlflush {id::123}} will be written to the
auxiliary file and in most cases that is quite okay, but if we have this:

\starttyping
\setuphead[title][expansion=yes]
\stoptyping

then we don't want the \type {#} to end up as hash because later on \TEX\
can get very confused about it because it sees some argument then in a
probably unexpected way. This is solved by escaping the hash like this:

\starttyping
About \Ux{23}3
\stoptyping

The \type {\Ux} command will convert its hexadecimal argument into a
character. Of course one then needs to typeset such a text under a \TEX\
character regime but that is normally the case anyway.

\stopsection

\stopchapter

\stopcomponent
