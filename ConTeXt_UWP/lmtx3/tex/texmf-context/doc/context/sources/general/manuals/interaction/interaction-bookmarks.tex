% language=uk

\environment interaction-style

\startcomponent interaction-bookmarks

\startchapter[title=Bookmarks]

Bookmarks are a sort of table of contents displayed by the viewer and as such
they take up extra space on the screen. You need to turn on interaction in
order to get bookmark data embedded in the document.

\starttyping
\placebookmarks
  [chapter,section,subsection,mylist]
  [chapter]
\stoptyping

A bookmark list is added to the document only when interaction is enabled. The
list in the first argument are bookmarked while the second argument specifies
what bookmark (sub)trees are opened. if you don't get what you expect, check your
document structure! Also, use the \type {\start}|-|\type {stop} alternatives.

Bookmarks are taken from the section title, but you can overload the title
as follows:

\starttyping
\startchapter[title=Foot,bookmark=food]
    ...
\stopchapter
\stoptyping

If you have a more complex typeset title you can also try:

\starttyping
\enabledirectives[references.bookmarks.preroll]
\stoptyping

From \MKII\ we inherit the option to overload the last set bookmark but the
previously mentioned approach is better.

\starttyping
\chapter {the first chapter}
\bookmark {the first bookmark}
\stoptyping

You can add entries to a bookmark list:

\starttyping
\bookmark[mylist]{whatever}
\stoptyping

This assumes that you have defined the list.

\showsetup {bookmark}

If you want to have the bookmark tab open when you start a document, you
can say:

\starttyping
\setupinteractionscreen[option=bookmark]
\stoptyping

There are only a few options that you can use. The \type {number} parameter can
be used to hide section numbers. The \type {sectionblock} parameter controls the
addition of section block entries, something that can be handy when you have
multiple section blocks with similar section titles. With \type {force} you force
an entry to the file, bypassing mechanisms that to be clever.

\showsetup{setupbookmark}

\stopchapter

\stopcomponent

