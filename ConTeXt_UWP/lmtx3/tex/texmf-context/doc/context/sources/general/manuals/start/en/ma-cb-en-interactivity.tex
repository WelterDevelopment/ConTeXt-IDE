\startcomponent ma-cb-en-interactivity

\enablemode[**en-us]

\project ma-cb

\startchapter[title=Interactive mode in electronic documents,reference=interactivity]

\startsection[title=Introduction]

\index[reader]{\READER}

Documents that are electronically available for consulting and displaying on a
computer screen are called interactive documents.

Interaction means that you can click on active areas and jump to the indicated
locations. For example if you consult a register you can click on a (active) page
number and you will jump to the corresponding page.

Interaction relates to:

\startitemize[packed]
\item active chapter numbers in the table of content
\item active page numbers in registers
\item active page numbers, chapter numbers and figure numbers in internal
      references to pages, chapters, figures etc. in the running text
\item active titles, page numbers, and chapter numbers in external references to
      other interactive documents
\item active menus as navigation tools
\item references to webpages and programs
\stopitemize

Interactivity depends on the program you use to view the interactive document. We
assume here that you will use \READER\ for viewing.

\CONTEXT\ is a very powerful system for producing electronic or interactive \PDF\
documents. However, only a few standard features are described in this chapter. As
the authors of this manual are planning to make all \CONTEXT\ related manuals
electronically (sources included) available, reverse engineering is one of the
options to become more acquainted with the possibilities of \CONTEXT.

Good examples of interactive documents are \CONTEXT\ presentations (see \in{chapter}[presentations]).
For more complex interactive \PDF\ documents with forms you should read the
\goto {Widgets} [ url (manual:widget) ]
manual.

\stopsection

\startsection[title=Interactive mode]

\index{interactive mode}

\Command{\tex{setupinteraction}}

The interactive mode is activated by:

\shortsetup{setupinteraction}

For example:

\startbuffer
\setupinteraction
  [state=start,
   color=green,
   style=bold]
\stopbuffer

\typebuffer

The hyper links are now generated automatically and the active words are
displayed in bold green.

The interactive document is considerably bigger (in MB's) than its paper cousin
because hyperlinks consume space. You will also notice that processing time
becomes longer. Therefore it is advisable to de-activate the interactive mode as
long as your document is under construction.

\stopsection

\startsection[title=Interaction within a document]

\index{interaction+internal}

\Command{\tex{in}}
\Command{\tex{at}}
\Command{\tex{goto}}

Earlier you have seen how to make a reference with \type{\in} and \type{\at}. You
may have wondered why you had to type \type{\in{chapter}[chap:introduction]}. In
the first place {\em chapter} and its corresponding chapter number will not be
separated at line breaking. In the second place the word {\em chapter} and its
number are typeset differently in the interactive mode. This gives the user a
larger clickable area.

\stopsection

\startsection[title=Interaction between documents]

\index{interaction+external}

\Command{\tex{from}}
\Command{\tex{useexternaldocument}}

It is possible to link one document to another. First you have to state that you
want to refer to another document. This is done by:

\shortsetup{useexternaldocument}

The first bracket pair must contain a logical name of the document, the second
pair the file name of the other document and the third pair is used for the title
of the document.

For refering to these other documents you can use:

\shortsetup{from}

The curly braces contain text and the brackets contain the reference.

Look at the example below.

\startbuffer
\useexternaldocument
   [hia][hasseltbook][Festivities in Hasselt]

Most tourist attractions are described in \from[hia].\crlf
A description of the \about[hia::euifeest] is found in \from[hia].\crlf
The eui||feest is described on \at{page}[hia::euifeest] in \from[hia].\crlf
See for more information \in{chapter}[hia::euifeest] in \from[hia].
\stopbuffer

\typebuffer

The \type{\useexternaldocument} is usually typed in the set up area of your input
file.

After processing your input file and the file \type{hasseltbook.tex},
you will have two \PDF\ documents. The references come out like this:

\start
\setupinteraction[state=start]
\getbuffer
\stop

For more information on cross referencing look at
\goto {\CONTEXT\ Magazine 1103} [ url (thisway:crossrefs)].

\stopsection

\startsection[title=Interaction with the world wide web]

In interactive mode there is one other command that has little meaning in the
paper version.

\shortsetup{goto}

The curly braces contain text, the brackets contain a reference (logical name or a
location).

\startbuffer
In \goto {Hasselt} [ url(http://www.stadindex.nl/plattegrond/hasselt) ]
all streets are build in a circular way.
\stopbuffer

\typebuffer

In the interactive document \goto{Hasselt}[ url(http://www.stadindex.nl/plattegrond/hasselt) ]
will be green and active. When you click the text you will jump to a map of Hasselt.

For a consistent definition of the urls there is the command:

\shortsetup{useURL}

The adress is defined with:

\startbuffer
\useURL
  [loc:cityplan]                                % id
  [http://www.stadindex.nl/plattegrond/hasselt] % adress
  []                                            % document
  []                                            % text
\stopbuffer

\typebuffer

The webadress is recalled by its logical name:
\type{\goto {Hasselt} [ url(loc:cityplan) ]}.

It is of good practice to define and maintain the urls in a separate
file.

\stopsection

\startsection[title=Buttons]

The command to define a button is:

\shortsetup{button}

The first bracket pair contains the setup keys, the curly brackets contain the
button text and the last bracket pair the destination.

\startbuffer
\useexternalsoundtrack
  [stranger][wayfaring_stranger.mp3]

\button{Website Hasselt} [ url(http://www.hasselt.nl) ]
\button{MSWord Document} [ program(hasselt.doc) ]
\button{Sound Clip}      [ StartSound{stranger} ]
\stopbuffer

\typebuffer

The first example results in a jump to a webpage, the second opens the file
\type{hasselt.doc} in \MSWORD\ and the third plays a tune. Note the use of the
\type{\useexternalsoundtrack} command.

\stopsection

\startsection[title=Menus]

You can define a menu with:

\shortsetup{startinteractionmenu}

And set it up with:

\shortsetup{setupinteractionmenu}

The first bracket pair is used for its name and the second pair for setting up
the menu.

A menu can be used in an interactive document. Below you can find a simple
example that you can copy to do some experimenting:

\startbuffer
\setuppapersize
  [S6][S6]

\setuplayout
  [header=0cm,          topspace=.5cm, backspace=2cm,
   margindistance=.5cm, margin=1cm,    rightmargin=0cm,
   edgedistance=.5cm,   rightedge=2cm, width=fit,
   height=13.8cm,       footer=1cm,    bottom=1cm]

\setupinteraction
  [state=start,         menu=on]

\setupinteractionmenu
  [bottom]
  [background=color,    backgroundcolor=gray, frame=off]

\startinteractionmenu[bottom]
\hfill
\startbut [content]       contents      \stopbut \quad
\startbut [index]         index         \stopbut \quad
\startbut [PreviousJump]  last location \stopbut \quad
\startbut [NextPage]      next page     \stopbut \quad
\startbut [CloseDocument] exit          \stopbut \quad
\stopinteractionmenu

\starttext

\startstandardmakeup
  \midaligned{\tfd Festivities in Hasselt}
\stopstandardmakeup

\completecontent

\startchapter[title=Introduction]
  An introduction.
\stopchapter

\startchapter[title=Kingsday]
  Something about Kingsday in Hasselt.\index{Kingsday}
\stopchapter

\startchapter[title=Hassailt]
  Something about Hassailt.\index{Hassailt}
\stopchapter

\startchapter[title=Euifeest,reference=euifeest]
  Something about the Euifeest.\index{Euifeest}
\stopchapter

\completeindex

\stoptext
\stopbuffer

\typebuffer

The definition of the \type{\startinteractionmenu} will produce a menu at the
bottom of every screen. The menu buttons contain the text {\em contents}, {\em
index}, {\em last location}, {\em next page} and {\em exit} with respectively the
following functions: jump to the table of contents, jump to the index, goto the
last location in the document, goto next page and close the document. The labels
to obvious destinations like \type{content} and \type{index} are predefined.
Other predefined destinations are \type{FirstPage}, \type{LastPage},
\type{NextPage} and \type{PreviousPage}.

An action like \type{CloseDocument} is necessary to make an electronic document
self containing. Other predefined actions you can use are \type{PrintDocument},
\type{SearchDocument} and \type{PreviousJump}. The meaning of these actions is
obvious.

\stopsection

\stopchapter

\stopcomponent
