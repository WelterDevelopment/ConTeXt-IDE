\startcomponent ma-cb-en-math

\enablemode[**en-us]

\project ma-cb

\startchapter[reference=formulas,title=Typesetting math]

\startsection[title=Introduction]

\index {math}

\TEX\ is {\em the} typesetting program for math. However, this is not the
extensive chapter on typesetting math you might expect. We advise you to do some
further reading on typesetting formulas in \TEX. See for example: \footnote{In
this introduction on typesetting math we relied on the booklet {\em \TEX niques}
by Arthur Samuel.}

\startitemize[packed]
\item {\em The \TeX Book} by D.E. Knuth
\item {\em The Beginners Book of \TeX} by S. Levy and R. Seroul
\stopitemize

\startsection[title=Typesetting math]

\index {math mode}
\index {display mode}
\index {text mode}

Normally different conventions are applied for typesetting normal text and math
text. These conventions are \quote{known} by \TEX\ and applied accordingly when
generating a document. We can rely on \TEX\ for delivering high quality math
output.

A number of conventions for math are:

\startitemize[n,packed]

\item Characters are typeset in $math\ italic$ (don't confuse this with the
      normal {\it italic characters} in a font).

\item Symbols like Greek characters ($\alpha$, $\chi$) and math symbols ($\leq$,
      $\geq$, $\in$) are used.

\item Spacing will differ from normal spacing.

\item Math expressions have a different alignment than that of the running text.

\item The sub and superscripts are downsized automatically, like in $a^{b}_{c}$.

\item Certain symbols have different appearances in the inline and display mode.

\stopitemize

When typesetting math you have to work in the so called math mode in which math
expressions can be defined by means of plain \TEX||commands.

Math mode has two alternatives: text mode and display mode. Math in text
mode is activated by \type{$} and \type{$}, while display mode is activated by
\type{$$} and \type{$$}. In \CONTEXT\ however, display mode is activated with
the \type{\start ... \stopformula} command pair to have more grip on vertical
spacing around the formula.

\startbuffer
The municipality of Hasselt covers an area of 42,05 \unit{Square Kilo
Meter}. Now, if you consider a circular area of this size with the
market place of Hasselt as the center point $M$ you can calculate its
diameter with ${{1}\over{4}} \pi r^2$.
\stopbuffer

\typebuffer

This will become:

\getbuffer

The many \type{{}} (grouping) in ${{1}\over{4}} \pi r^2$ are essential for
separating operations in the expression. If you omit the outer curly braces like
this: \type{${1}\over{4} \pi r^2$}, you would get a non desired result:
${1}\over{4} \pi r^2$.

The letters and numbers are typeset in three different sizes: text size $a+b$,
script size $\scriptstyle a+b$ and scriptscript size $\scriptscriptstyle a+b$.
These can be influenced by the commands \type{\scriptstyle} and
\type{\scriptscriptstyle}.

Symbols like $\int$ and $\sum$ will have a different form in text and display
mode. If we type \type {$\sum_{n=1}^{m}$} or \type {$\int_{-\infty}^{+\infty}$}
we will get {$\sum_{n=1}^{m}$} and {$\int_{-\infty}^{+\infty}$}. But when you
type:

\startbuffer
\startformula
  \sum_{n=1}^{m} \quad {\rm and} \quad \int_{-\infty}^{+\infty}
\stopformula
\stopbuffer

\typebuffer

to get displaymode you get:

\getbuffer

With the commands \type {\nolimits} and \type{\limits} you can influence the
appearances of \type{\sum} and \type{\int}:

\startbuffer
\startformula
  \sum_{n=1}^{m}\nolimits
  \quad {\rm and} \quad
  \int_{-\infty}^{+\infty}\limits
\stopformula
\stopbuffer

\typebuffer

which will result in:

\getbuffer

For typesetting fractions there is the command \type {\over}. In \CONTEXT\ you
can use the alternative \type {\frac}. For ${\frac{a}{1+b}}+c$ we type for
instance \type {${\frac{a}{1+b}}+c$}.

Other commands to put one thing above the other, are:

\startbuffer[atop]
${a} \atop {b}$
\stopbuffer
\startbuffer[choose]
${n+1} \choose {k}$
\stopbuffer
\startbuffer[brack]
${m} \brack {n}$
\stopbuffer
\startbuffer[brace]
${m} \brace {n-1}$
\stopbuffer

\starttabulate[|l|l|l|l|]
\NC \type {\atop}
\NC \typebuffer[atop]
\NC \mathstrut\getbuffer[atop]
\NC
\NC\NR
\NC \type {\choose}
\NC \typebuffer[choose]
\NC
\NC \mathstrut\getbuffer[choose]
\NC\NR
\NC \type {\brack}
\NC \typebuffer[brack]
\NC \mathstrut\getbuffer[brack]
\NC
\NC\NR
\NC \type {\brace}
\NC \typebuffer[brace]
\NC
\NC \mathstrut\getbuffer[brace]
\NC\NR
\stoptabulate

\TEX\ can enlarge delimiters like (~) and $\{~\}$ automatically if the left and
right delimiter is preceeded by the commands \type {\left} and \type {\right}
respectively. If you type:

\startbuffer
\startformula
  1+\left(\frac{1}{1-x^{x-2}}\right)^3
\stopformula
\stopbuffer

\typebuffer

you will get:

\getbuffer

Sub and superscripts are invoked by \quote {\type{_}} and \quote {\type{^}}. They
have effect on the next first character so grouping with $\{$~$\}$ is necessary
in case of multi character sub and superscripts.

In certain situations the delimiters can be preceeded by \type{\bigl},
\type{\Bigl}, \type{\biggl} and \type{\Biggl} and their right counterparts. Even
bigger delimiters can be produced by placing \type{\left} and \type{\right} in a
\type{\vbox} construction. When we type a senseless expression like:

\startbuffer
\startformula
  \left(\vbox to 16pt{}x^{2^{2^{2^{2}}}}\right)
\stopformula
\stopbuffer

\typebuffer

we get:

\getbuffer

In display mode the following delimiters will work in the automatic enlargement
mechanism:

\starttabulate[|l|l|l|l|l|l|l|l|]
\NC \type{\lfloor}      \NC $\lfloor$
\NC \type{\langle}      \NC $\langle$
\NC \type{\vert}        \NC $\vert$
\NC \type{\downarrow}   \NC $\downarrow$
\NC\NR
\NC \type{\rfloor}      \NC $\rfloor$
\NC \type{\rangle}      \NC $\rangle$
\NC \type{\Vert}        \NC $\Vert$
\NC \type{\Downarrow}   \NC $\Downarrow$
\NC\NR
\NC \type{\lceil}       \NC $\lceil$
\NC \type{/}            \NC $/$
\NC \type{\uparrow}     \NC $\uparrow$
\NC \type{\updownarrow} \NC $\updownarrow$
\NC\NR
\NC \type{\rceil}       \NC $\rceil$
\NC \type{\backslash}   \NC $\backslash$
\NC \type{\Uparrow}     \NC $\Uparrow$
\NC \type{\Updownarrow} \NC $\Updownarrow$
\NC\NR
\stoptabulate

In display mode we should typeset only one fraction and otherwise switch to the
\type{a/b} notation. To get:

\startformula
  a_0 + {\frac{a}{a_1 + \frac{1}{a_2}}}
\stopformula

we will not type:

\startbuffer
\startformula
  a_0+{\frac{a}{a_1+\frac{1}{a_2}}}
\stopformula
\stopbuffer

\typebuffer

but prefer:

\startbuffer
\startformula
  a_0 + {\frac{a}{a_1 + 1/a_2}}
\stopformula
\stopbuffer

\typebuffer

to obtain:

\getbuffer

In addition we could also use the command \type{\displaystyle}. If we would type:

\startbuffer
\startformula
  a_0 + {\frac{a}{a_1 + \frac{1}{\strut \displaystyle a_2}}}
\stopformula
\stopbuffer

\getbuffer

we will get:

\getbuffer

Below we demonstrate the commands \type{\matrix}, \type{\pmatrix}, \type{\ldots},
\type{\cdots} and \type{\cases} without any further explanation.

\startbuffer[a]
\startformula
\stopbuffer

\startbuffer[c]
\stopformula
\stopbuffer

\startbuffer[b]
  A=\left(\matrix{x-\lambda & 1         & 0         \cr
                  0         & x-\lambda & 1         \cr
                  0         & 0         & x-\lambda \cr}\right)
\stopbuffer

\typebuffer[a,b,c] \startformula\getbuffer[b]\stopformula

\startbuffer[b]
  A=\left|\matrix{x-\mu& 1     & 0     \cr
                  0    & x-\mu & 1     \cr
                  0    & 0     & x-\mu \cr}\right|
\stopbuffer

\typebuffer[a,b,c] \startformula\getbuffer[b]\stopformula

\startbuffer[b]
  A=\pmatrix{a_{11} & a_{12} & \ldots & a_{1n} \cr
             a_{21} & a_{22} & \ldots & a_{2n} \cr
             \vdots & \vdots & \ddots & \vdots \cr
             a_{m1} & a_{m2} & \ldots & a_{mn} \cr}
\stopbuffer

\typebuffer[a,b,c] \startformula\getbuffer[b]\stopformula

\startbuffer[b]
  A=\pmatrix{a_{11} & a_{12} & \ldots & a_{1n} \cr
             a_{21} & a_{22} & \ldots & a_{2n} \cr
             \vdots & \vdots & \ddots & \vdots \cr
             a_{m1} & a_{m2} & \ldots & a_{mn} \cr}
\stopbuffer

\typebuffer[a,b,c] \startformula\getbuffer[b]\stopformula

\startbuffer[b]
  |x|=\cases{ x, & if $x\geq0$; \cr
             -x, & otherwise    \cr}
\stopbuffer

\typebuffer[a,b,c] \startformula\getbuffer[b]\stopformula

To typeset normal text in a math expression we have to consider the following.
First a space is not typeset in math mode so we have to enforce one with
\type{ \ } (backslash). Second we have to indicate a font switch, because the text should
not appear in $math\ italic$ but in the actual font. So in \CONTEXT\ we have to
type:

\startbuffer
\startformula
  x^3+{\tf lower\ order\ terms}
\stopformula
\stopbuffer

\typebuffer

to get:

\getbuffer

The math functions like $\sin$ and $\tan$ that have to be typeset in the actual
font are predefined functions in \TEX:

\starttabulate[|l|l|l|l|l|l|l|l|]
\NC \type{\arccos} \NC \type{\cos} \NC \type{\csc} \NC \type{\exp} \NC \type{\ker} \NC \type{\limsup} \NC \type{\min} \NC \type{\sinh} \NC\NR
\NC \type{\arcsin} \NC \type{\cosh} \NC \type{\deg} \NC \type{\gcd} \NC \type{\lg} \NC \type{\ln} \NC \type{\Pr} \NC \type{\sup} \NC\NR
\NC \type{\arctan} \NC \type{\cot} \NC \type{\det} \NC \type{\hom} \NC \type{\lim} \NC \type{\log} \NC \type{\sec} \NC \type{\tan} \NC\NR
\NC \type{\arg} \NC \type{\coth} \NC \type{\dim} \NC \type{\inf} \NC \type{\liminf} \NC \type{\max} \NC \type{\sin} \NC \type{\tanh} \NC\NR
\stoptabulate

If we type the sinus or limit function:

\startbuffer
\startformula
  \sin 2\theta=2\sin\theta\cos\theta
  \quad {\tf or} \quad
  \lim_{x\to0}{\frac{\sin x}{x}}=1
\stopformula
\stopbuffer

\typebuffer

we get:

\getbuffer

Alignment in math expressions may need special attention. In multi line
expressions we sometimes need alignment at the \quote {$=$} sign. This is done by
the command \type{\eqalign}. If we type:

\startbuffer
\startformula
  \eqalign{
    ax^2+bx+c &= 0                                \cr
    x         &= \frac{-b \pm \sqrt{b^2-4ac}}{2a} \cr}
\stopformula
\stopbuffer

\typebuffer

we get:

\getbuffer

Sometimes alignment at more than one location is wanted. Watch the second line in
the next example and see how it is defined:

\startbuffer
\startformula
  \eqalign{
    ax+bx+\cdots+yx+zx &         = x(a +b+ \cdots \cr
                       &\phantom{= x(a~}+y+z)     \cr
                       &         = y              \cr}
\stopformula
\stopbuffer

\typebuffer

This results in:

\getbuffer

Next to the command \type{\phantom} there are \type{\hphantom} without height and
depth and \type{\vphantom} without width.

You can rely on \TEX\ for spacing within a math expression. In some situations,
however you may want to influence spacing. This is done by:

\starttabulate[|l|r|]
\NC \type{\!} \NC $-\frac{1}{6}$\type{\quad} \NC\NR
\NC \type{\,} \NC $\frac{1}{6}$\type{\quad}  \NC\NR
\NC \type{\>} \NC $\frac{2}{9}$\type{\quad}  \NC\NR
\NC \type{\;} \NC $\frac{5}{18}$\type{\quad} \NC\NR
\stoptabulate

These \quote {spaces} are related to \type {\quad} that stands for the width of
the capital \quote{M}.

The use of the command \type{\prime} speaks for itself. For example if would want
$y_1^\prime+y_2^{\prime\prime}$ you should type
\type{$y_1^\prime+y_2^{\prime\prime}$}.

An expression like $\root 3 \of {x^2+y^2}$ is obtained by \type{$\root 3 \of
{x^2+y^2}$}.

At the end of this section we point to the command \type{\mathstrut} which we can
use to enforce consistency, for example within the root symbol. With
\type{$\sqrt{\mathstrut a}+\sqrt{\mathstrut d}+\sqrt{\mathstrut y}$} we will get
$\sqrt{\mathstrut a}+\sqrt{\mathstrut d}+\sqrt{\mathstrut y}$ in stead of
$\sqrt{a}+\sqrt{d}+\sqrt{y}$.

See \in{appendix}[overviews] for a complete overview of math commands.

\stopsection

\startsection[title=Placing formulas]

\index{formula}

\Command{\tex{placeformula}}
\Command{\tex{startformula}}
\Command{\tex{setupformulas}}

You can typeset numbered formulas with:

\shortsetup{placeformula}
\shortsetup{startformula}

Two examples:

\startbuffer
\placeformula[formula:aformula]
  \startformula
     y=x^2
  \stopformula

\placeformula
  \startformula
    \int_0^1 x^2 dx
  \stopformula
\stopbuffer

\typebuffer

\getbuffer

The command \type{\placeformula} handles spacing around the formulas and the
numbering. The bracket pair is optional and is used for referencing and to switch
numbering on and off.

\startbuffer
\placeformula[first one]
\startformula
  y=x^2
\stopformula

\placeformula[middle one]
\startformula
  y=x^3
\stopformula

\placeformula[last one]
\startformula
  y=x^4
\stopformula
\stopbuffer

\getbuffer

\in{Formula}[middle one] was typed like this:

\startbuffer
\placeformula[middle one]
  \startformula
     y=x^3
  \stopformula
\stopbuffer

\typebuffer

The lable \type{[middle one]} is used for refering to this formula. Such a
reference is made with \type{\in{formula}[middle one]}.

If no numbering is required you type:

\type{\placeformula[-]}

Numbering of formulas is set up with \type{\setupnumbering}. In this manual
numbering is set up with \type{\setupnumbering[way=bychapter]}. This means that
the chapter number preceeds the formula number and numbering is reset with each
new chapter. For reasons of consistency the tables, figures, intermezzi etc. are
numbered in the same way. Therefore you use \type{\setupnumbering} in the set up
area of your input file.

Formulas can be set up with:

\shortsetup{setupformulae}

\stopsection

\stopchapter

\stopcomponent

